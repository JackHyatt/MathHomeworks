% !TeX program = lualatex

\documentclass[14pt]{extarticle}



\usepackage[margin=1in]{geometry} 
\usepackage{amsmath,amsthm,amssymb}
\usepackage{MnSymbol}
\usepackage{graphicx}
\usepackage{bm}
\usepackage[normalem,normalbf]{ulem}
\usepackage{algorithm} 
\usepackage{algpseudocode} 
\usepackage{multirow}
\usepackage{rotating}
\usepackage{therefore}

\usepackage{tikz}
\usetikzlibrary{shapes.multipart}
\usetikzlibrary{shapes.symbols}

\usetikzlibrary{graphs,graphdrawing,graphs.standard,quotes}
\usegdlibrary{circular,force,layered,routing}
\tikzset{
	graphs/simpleer/.style={
		nodes={draw,circle, blue, left color=blue!20, text=black, inner sep=1pt},
		node distance=2.5cm, nodes={minimum size=2em}
	},
	every loop/.style={},
}

\newcommand*\circled[1]{\tikz[baseline=(char.base)]{
		\node[shape=circle,draw,inner sep=2pt] (char) {#1};}}

\newcommand{\m}{\medskip\\}
\newcommand{\N}{\mathbb{N}}
\newcommand{\Z}{\mathbb{Z}}
\newcommand{\R}{\mathbb{R}}
\newcommand{\bbs}{\textbackslash\textbackslash\space}
\newcommand{\bs}{\textbackslash\space}
\newcommand{\la}{\enskip\land\enskip}
\newcommand{\lo}{\enskip\lor\enskip}
\newcommand{\comp}[1]{#1^\mathsf{c}}
\newcommand{\micdrop}{\qed}
\newcommand{\contra}{\begin{tikzpicture}
		\node[starburst, draw, minimum width=3cm, minimum height=2cm,line width=1.5pt,red,fill=yellow,scale=.5]
		{BOOM, A CONTRADICTION!!!};
\end{tikzpicture}}

\renewcommand{\qedsymbol}{$\blacksquare$}

\DeclareMathOperator{\lcm}{lcm}

\newtheorem{theorem}{Theorem}

\newenvironment{exercise}[2][Exercise]{\begin{trivlist}
		\item[\hskip \labelsep {\bfseries #1}\hskip \labelsep {\bfseries #2.}]}{\end{trivlist}}

\setlength\parindent{24pt}

\makeatletter
\renewcommand*\env@matrix[1][*\c@MaxMatrixCols c]{%
	\hskip -\arraycolsep
	\let\@ifnextchar\new@ifnextchar
	\array{#1}}
\makeatother
\setlength\parindent{24pt}


\begin{document}
	
	% --------------------------------------------------------------
	%                         Start here
	% --------------------------------------------------------------
	
	
	\title{Homework 5 (Due Sept 22, 2023)}
	\author{Jack Hyatt\\ %replace with your name
		MATH 554 - Analysis I - Fall 2023} 
	
	\maketitle
	
	Justify all of your answers completely.\\
	
	
	\medskip 
	
	\begin{enumerate}
		\item Prove every open ball $B(a,r)$ is open.
		\begin{proof}
			Let $b \in B(a,r)$. Then $d(a,b) < r$. Let $\rho = r-d(a,b)$.\\
			Let $b' \in B(b,\rho)$. Then $d(b,b') < \rho \implies d(b,b') + d(a,b) < r$. Since $d(a,b') \leq d(a,b) + d(b,b')$ by the triangle inequality, we get $d(a,b') < r$. So $b' \in B(a,r)$.\\
			So $B(b,\rho) \subseteq B(a,b)$, which means $B(a,r)$ is open.
		\end{proof}
		
		\item Prove for any $a \in E$ and $r > 0$ the set $U = \{x \in E : x \nin \bar{B}(a,r)\} = \{x \in E : d(x,a) > r\}$ is open.
		\begin{proof}
			Let $b \in U$. Then $d(a,b) > r > 0$. Let $\rho = d(a,b)-r$.\\
			Let $b' \in B(b,\rho)$, so $d(b,b') < \rho$. By rearranging the triangle inequality, $d(a,b') \geq d(a,b) - d(b,b')$.\\
			Then $d(b,b') < \rho \implies d(a,b)-d(b,b') > r$. Using the triangle inequality, we get $d(a,b') > r$. So $b' \in U$.\\
			So $B(b,\rho) \subseteq U$, which means $U$ is open.
		\end{proof}
		
		\item Let $\{U_\alpha : \alpha \in A\}$ be a possibly infinite collection of open subsets of $E$. Prove that the union 
		\[U \coloneq \bigcup_{\alpha \in A} U_\alpha\] is open.
		\begin{proof}
			Let $a \in U$. Then $a \in U_\alpha$ for some $\alpha \in A$. Since $U_\alpha$ is open, then $\exists r$ s.t. $B(a,r) \subseteq U_\alpha$. Since $U$ is comprised of the unions of sets, $U_\alpha \subseteq U$. So $B(a,r) \subseteq U$. This means $U$ is open.
		\end{proof}
		
		\item Let $U_1,\ldots, U_n \subseteq E$ be a finite collection of open subsets of $E$. Prove that the intersection 
		\[U = U_1 \cap \ldots \cap U_n\] is open.
		\begin{proof}
			Let $a \in U$. Then $\forall j \in [n], a \in U_j$. So $\forall j \in [n], \exists r_j > 0$ s.t. $ B(a,r_j) \subseteq U_j$. Let $r$ be the $\min(r_1,\ldots,r_n)$. Then $B(a,r) \subseteq B(a,r_j)$ for every $j$. So $B(a,r) \subseteq U_j$ for every $j$, which means $B(a,r) \subseteq U$. So $U$ is open.
		\end{proof}
		
		\item Let $U_n = (-1/n,1/n)$ in $\R$. Show \[U = \bigcap_{n=1}^{\infty}U_n = \{0\}\] and therefore the intersection is not open.
		\begin{proof}
			$U_n = (-1/n,1/n)$ is equivalent to $U_n = B(0,1/n)$.\\
			BWOC, let $x \in U$ not be $0$. WLOG, let $x$ be positive.\\
			So then $\forall n\in\N,\medspace x<1/n$. This violates Archimedes’ axiom (small version). So then $x$ cannot be positive (and not negative since same can be said for $-x$).\\
			$x = 0$ does work since $1/n$ will always be a nonzero number.\\
			So $U = \{0\}$. Since it is a singleton set, there is no way for it to be open, trust me.
		\end{proof}
	\end{enumerate}
\end{document}