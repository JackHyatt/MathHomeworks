% !TeX program = lualatex

\documentclass[14pt]{extarticle}



\usepackage[margin=1in]{geometry} 
\usepackage{amsmath,amsthm,amssymb}
\usepackage{MnSymbol}
\usepackage{graphicx}
\usepackage{bm}
\usepackage[normalem,normalbf]{ulem}
\usepackage{algorithm} 
\usepackage{algpseudocode} 
\usepackage{multirow}
\usepackage{rotating}
\usepackage{therefore}

\usepackage{tikz}
\usetikzlibrary{shapes.multipart}
\usetikzlibrary{shapes.symbols}

\usetikzlibrary{graphs,graphdrawing,graphs.standard,quotes}
\usegdlibrary{circular,force,layered,routing}
\tikzset{
	graphs/simpleer/.style={
		nodes={draw,circle, blue, left color=blue!20, text=black, inner sep=1pt},
		node distance=2.5cm, nodes={minimum size=2em}
	},
	every loop/.style={},
}

\newcommand*\circled[1]{\tikz[baseline=(char.base)]{
		\node[shape=circle,draw,inner sep=2pt] (char) {#1};}}

\newcommand{\m}{\medskip\\}
\newcommand{\N}{\mathbb{N}}
\newcommand{\Z}{\mathbb{Z}}
\newcommand{\R}{\mathbb{R}}
\newcommand{\bbs}{\textbackslash\textbackslash\space}
\newcommand{\bs}{\textbackslash\space}
\newcommand{\la}{\enskip\land\enskip}
\newcommand{\lo}{\enskip\lor\enskip}
\newcommand{\comp}[1]{#1^\mathsf{c}}
\newcommand{\micdrop}{\qed}
\newcommand{\contra}{\begin{tikzpicture}
		\node[starburst, draw, minimum width=3cm, minimum height=2cm,line width=1.5pt,red,fill=yellow,scale=.5]
		{BOOM, A CONTRADICTION!!!};
\end{tikzpicture}}

\renewcommand{\qedsymbol}{$\blacksquare$}

\DeclareMathOperator{\lcm}{lcm}

\newtheorem{theorem}{Theorem}

\newenvironment{exercise}[2][Exercise]{\begin{trivlist}
		\item[\hskip \labelsep {\bfseries #1}\hskip \labelsep {\bfseries #2.}]}{\end{trivlist}}

\setlength\parindent{24pt}

\makeatletter
\renewcommand*\env@matrix[1][*\c@MaxMatrixCols c]{%
	\hskip -\arraycolsep
	\let\@ifnextchar\new@ifnextchar
	\array{#1}}
\makeatother
\setlength\parindent{24pt}


\begin{document}
	
	% --------------------------------------------------------------
	%                         Start here
	% --------------------------------------------------------------
	
	
	\title{Homework 13 (Due Nov 10, 2023)}
	\author{Jack Hyatt\\ %replace with your name
		MATH 554 - Analysis I - Fall 2023} 
	
	\maketitle
	
	Justify all of your answers completely.\\
	
	
	\medskip 
	
	\begin{enumerate}
		\item
		\begin{enumerate}
			\item Show that the function \[f(x) = \frac{x}{1+x^2}\] is Lipschitz on the closed interval $[-b,b]$ for any $b>0$.
			\begin{proof}
				Let $x,y \in [-b,b]$.\\
				Then 
				\[|f(x) - f(y)| = \left|\frac{x}{1+x^2} - \frac{y}{1+y^2}\right| = \left|\frac{x+xy^2-y-yx^2}{(1+x^2)(1+y^2)}\right|\]
				\[=|x-y|\left|\frac{1-xy}{x^2y^2+x^2+y^2+1}\right| \leq |x-y||1-xy| \leq |x-y||1+b^2|\]
				Letting $1+b^2$ be the Lipschitz constant shows that $f(x)$ is Lipschitz on $[-b,b]$.
			\end{proof}
			\item Use this to give a detailed $N,\epsilon$ proof that if $\langle x_n \rangle_{n=1}^\infty$ us a sequence of real numbers with $\lim_{n\rightarrow\infty}x_n=L$ that
			\[\lim_{n\rightarrow\infty}\frac{x_n}{x^2_n+1} = \frac{L}{L^2+1}\]
			\begin{proof}
				Assume $\lim_{n\rightarrow\infty} x_n = L$.\\
				Then $\exists N$ s.t. $n>N \implies |x_n - L| < \frac{\epsilon}{b^2+1}$.\\
				Let $f(x)$ be on the interval $[-L,L]$, $\epsilon>0$, and $n>N$.\\
				Then 
				\[|f(x) - f(L)| = \left|\frac{x_n}{x_n^2+1} - \frac{L}{L^2+1}\right| \leq (b^2+1)|x_n - L| < (b^2+1) \frac{\epsilon}{b^2+1}\]
				So then 
				\[\left|\frac{x_n}{x_n^2+1} - \frac{L}{b^2+1}\right| < \epsilon\]
				which proves the limit.
			\end{proof}
		\end{enumerate}
	
		\item Let $(A,d_A)$ and $(B,d_B)$ be metric spaces. Let $E \coloneq A \times B$ and $d$ on $E$ by
		\[d((a_1,b_1),(a_2,b_2)) = d_A(a_1,a_2)+d_B(b_1,b_2)\]
		\begin{enumerate}
			\item Prove this is a metric on $E = A \times B$
			\begin{proof}
				Clearly $d$ is symmetric.\\
				Since $d_A$ and $d_B$ are nonnegative, $d=d_A+d_B \geq 0$.\\
				$d((a_1,b_1),(a_2,b_2))=0 \iff d_A(a_1,a_2)+d_B(b_1,b_2)=0 \iff d_A(a_1,a_2)=0 \text{ and } d_B(b_1,b_2)=0 \iff a_1=a_2 \text{ and } b_1=b_2$.\\
				Now to show triangle inequality:
				\[d((a_1,b_1),(a_2,b_2)) = d_A(a_1,a_2)+d_B(b_1,b_2)\]\[\leq d_A(a_1,a')+d_A(a',a_2)+d_B(b_1,b')+d_B(b',b_2)\]
				\[= d_A(a_1,a')+d_B(b_1,b')+d_A(a',a_2)+d_B(b',b_2)\]
				\[= d((a_1,b_1),(a',b'))+d((a',b'),(a_2,b_2))\]
				So all 4 requirements hold for $d$ to be a metric on $E$.
			\end{proof}
			\item Prove that if $A$ and $B$ are complete, then so is $E$.
			\begin{proof}
				Assume both $A$ and $B$ are complete. Then every Cauchy sequence in $A$ and $B$ will converge.\\
				It has been shown previously that if $\lim_{n\rightarrow\infty} x_n = x$ and $\lim_{n\rightarrow\infty} y_n = y$, then $\lim_{n\rightarrow\infty} (x_n,y_n) = (x,y)$.\\
				Let $\langle(a_n,b_n) \rangle_{n=1}^\infty$ be a Cauchy sequence in $E$.\\
				So then $\forall\epsilon>0$, $\exists N$ s.t. $m,\ell>N \implies d((a_m,b_m),(a_\ell,b_\ell)) < \epsilon$.
				Since $d = d_A+d_B$, then $d_A(a_m,a_\ell),d_B(b_m,b_\ell) \leq d((a_m,b_m),(a_\ell,b_\ell))$.\\
				So $d_A(a_m,a_\ell),d_B(b_m,b_\ell) < \epsilon$.\\
				So then $\langle a_n \rangle_{n=1}^\infty$ and $\langle b_n \rangle_{n=1}^\infty$ are Cauchy sequences in $A$ and $B$ respectively.\\
				So then $\langle a_n \rangle_{n=1}^\infty$ and $\langle b_n \rangle_{n=1}^\infty$ converge since $A$ and $B$ are complete.\\
				So then $\langle(a_n,b_n) \rangle_{n=1}^\infty$ also converges, meaning $E$ is also complete.
			\end{proof}
			\item Prove that if $A$ and $B$ are both sequentially compact, then so is $E$.
			\begin{proof}
				Assume both $A$ and $B$ are seq. compact. Then every sequence in $A$ and $B$ will have a convergent subsequence.\\
				Let $\langle(a_n,b_n) \rangle_{n=1}^\infty$ be a sequence in $E$.\\
				So then $\langle a_n \rangle_{n=1}^\infty$ has a convergent subsequence $\langle a_{n_j} \rangle_{j=1}^\infty$.\\
				We can't say that $\langle b_{n_j} \rangle_{j=1}^\infty$ converges, but since $B$ is seq. compact, can say that there exists $\langle b_{n_{j_k}} \rangle_{k=1}^\infty$ that does converge.\\
				$\langle a_{n_{j_k}} \rangle_{k=1}^\infty$ is a subsequence of $\langle a_{n_j} \rangle_{j=1}^\infty$, which already converged. So $\langle a_{n_{j_k}} \rangle_{k=1}^\infty$ converges too.\\
				So then $\langle (a_{n_{j_k}},b_{n_{j_k}}) \rangle_{k=1}^\infty$ is a convergent subsequence of $\langle(a_n,b_n) \rangle_{n=1}^\infty$, proving $E$ is complete.
			\end{proof}
		\end{enumerate}
	
		\item Let $(E,d)$ and $(E',d')$ be metric spaces and let $f:E\rightarrow E'$ be Lipschitz. Prove if $V \subseteq E'$ is an open set, then \[U \coloneq f^{-1}[V] = \{p\in E : f(p) \in V\}\]
		is also open.
		\begin{proof}
			Let $p \in U$. Then $f(p) \in V$.\\
			Since $V$ is open, that means $\exists r>0$ s.t. $B(f(p),r) \subseteq V$ by definition of open.\\
			Let $M$ be the Lipschitz constant for $f$. Let $\delta = r/M$.\\
			If $q\in B(p,\delta)$, then $d(p,q)<r/M \implies Md(p,q)<r$.\\
			Since $d'(f(p),f(q))\leq Md(p,q)$, we know $d'(f(p),f(q))<r$ which means $f(q) \in B(f(p),r) \subseteq V$.\\
			So then $q \in B(p,\delta) \implies f(q) \in V \implies q \in U$.\\
			So then $B(p,\delta) \subseteq U$, which means $U$ is open by definition of open.
		\end{proof}
		
		\item Let $\R^2$ have its usual metric.
		\begin{enumerate}
			\item Let $\vec{a},\vec{b} \in \R^2$. Prove the map $f:\R\rightarrow\R^2$ defined by $f(t) = (1-t)\vec{a} + t\vec{b}$ is Lipschitz.
			\begin{proof}
				Let $x,y\in\R$. Then
				\[||f(x)-f(y)|| = ||(1-x)\vec{a}+x\vec{b}-(1-y)\vec{a}-y\vec{b}||\]
				\[= ||(y-x)\vec{a}+(x-y)\vec{b}|| = |x-y|\cdot||\vec{b}-\vec{a}||\]
				Since $||\vec{b}-\vec{a}||\geq0$, $f$ is Lipschitz.
			\end{proof}
			
			\item For $\vec{a},\vec{b} \in \R^2$ define the segment with endpoints $\vec{a}$ and $\vec{b}$ as 
			\[[\vec{a},\vec{b}] = \{(1-t)\vec{a}+t\vec{b} : 0 \leq t \leq 1\}\]
			Prove $[\vec{a},\vec{b}]$ is connected.
			\begin{proof}
				BWOC, let $[\vec{a},\vec{b}] = A \cupdot B$ be a disconnection.\\
				Let $A_0 = \{t \in [0,1] : (1-t)\vec{a}+t\vec{b} \in A\}$ and similarly for $B_0$.\\
				We can rewrite $A_0$ as $\{t \in [0,1] : f(t) \in A\}$ and similarly for $B_0$.\\
				We know that $f$ is Lipschitz from $(a)$ and since both $A$ and $B$ are open (know from them being a disconnection), Problem 2 tells us that the preimage of $A$ and $B$ through $f$ is also open. So $A_0$ and $B_0$ are both open.\\
				Since $A$ and $B$ are disjoint, so will $A_0$ and $B_0$ since taking the preimage preserves that property.\\
				Obviously $A_0 \neq \emptyset \neq B_0$ and $A_0 \cup B_0 = [0,1]$.\\
				So then $A_0 \cupdot B_0$ is a disconnection of $[0,1]$, which is a contradiction since closed intervals of the reals are connected.\\
				So then the disconnection $A \cupdot B$ cannot exist, making $\vec{a},\vec{b}]$ connected.
			\end{proof}
		\end{enumerate}
	
		\item Let $E$ be a metric space.
		\begin{enumerate}
			\item Show that if $E$ is compact, then any any collection, $\FF$, of closed subsets of $E$ with the finite intersection property has nonempty intersection. That is if $\FF$ has the finite intersection property, then \[\bigcap\FF\neq\emptyset\]
			\begin{proof}
				BWOC, assume $\FF$ has the finite intersection property and $\bigcap\FF=\emptyset$.\\
				Let $\U = \{F^c : F \in \FF\}$.\\
				Since $F$ is closed $\forall F \in \FF$, then $F^c$ is open.\\
				Since $\bigcap\FF=\emptyset$, then there does not exist any element that is in all $F\in\FF$. So then $\forall x\in E,\ \exists F\in\FF$ s.t. $x\in F^c$.\\
				So then $\U$ is an open cover of $E$.\\
				Since $E$ is compact, then $\exists\ \{F_1^c,\ldots,F_k^c\}$ that is a finite subcover of $E$.\\
				Since $E = \bigcup_{i=1}^k F_i^c$, then $E^c = (\bigcup_{i=1}^k F_i^c)^c \implies \emptyset = \bigcap_{i=1}^k F_i$.\\
				But that contradicts that $\FF$ has the finite intersection property, so $\bigcap\FF\neq\emptyset$.
			\end{proof}
		
			\item Conversely, show if every collection of closed sets, $\FF$, of $E$ with the finite intersection property has nonempty intersection, then $E$ is compact.
			\begin{proof}
				Let $\U$ be an open cover of $E$. BWOC, assume $\U$ does not have a finite subcover.\\
				Let $\FF\coloneq\{U^c : U \in \U\}$.\\
				Since $U$ are open sets, then $U^c$ are closed, meaning $\FF$ is a collection of closed sets.\\
				Since no finite subset of $\U$ can cover $E$, that means for every finite subset $\U_0\subseteq\U$ there is an element $x\in E$ s.t. $\forall U\in\U_0,\ x\nin U$. So then $\forall U\in\U_0,\ x\in U^c$.\\
				So then for any $\FF_0 = \U_0$, there exists an element $x \in E$ s.t. $x\in \bigcap\FF_0$. This means $\FF$ has the finite intersection property.\\
				Then by our assumption in the problem, $\bigcap\FF\neq\emptyset$. So there is an element in $E$ that is in every element of $\FF$, meaning there is an element of $E$ that is not in every element of $U$, which contradicts that $U$ is an open cover of $E$.\\
				So then $U$ must have a finite sub cover, meaning $E$ is compact.
			\end{proof}
		\end{enumerate}
	\end{enumerate}
\end{document}