% !TeX program = lualatex

\documentclass[14pt]{extarticle}



\usepackage[margin=1in]{geometry} 
\usepackage{amsmath,amsthm,amssymb}
\usepackage{MnSymbol}
\usepackage{graphicx}
\usepackage{bm}
\usepackage[normalem,normalbf]{ulem}
\usepackage{algorithm} 
\usepackage{algpseudocode} 
\usepackage{multirow}
\usepackage{rotating}
\usepackage{therefore}

\usepackage{tikz}
\usetikzlibrary{shapes.multipart}
\usetikzlibrary{shapes.symbols}

\usetikzlibrary{graphs,graphdrawing,graphs.standard,quotes}
\usegdlibrary{circular,force,layered,routing}
\tikzset{
	graphs/simpleer/.style={
		nodes={draw,circle, blue, left color=blue!20, text=black, inner sep=1pt},
		node distance=2.5cm, nodes={minimum size=2em}
	},
	every loop/.style={},
}

\newcommand*\circled[1]{\tikz[baseline=(char.base)]{
		\node[shape=circle,draw,inner sep=2pt] (char) {#1};}}

\newcommand{\m}{\medskip\\}
\newcommand{\N}{\mathbb{N}}
\newcommand{\Z}{\mathbb{Z}}
\newcommand{\R}{\mathbb{R}}
\newcommand{\bbs}{\textbackslash\textbackslash\space}
\newcommand{\bs}{\textbackslash\space}
\newcommand{\la}{\enskip\land\enskip}
\newcommand{\lo}{\enskip\lor\enskip}
\newcommand{\comp}[1]{#1^\mathsf{c}}
\newcommand{\micdrop}{\qed}
\newcommand{\contra}{\begin{tikzpicture}
		\node[starburst, draw, minimum width=3cm, minimum height=2cm,line width=1.5pt,red,fill=yellow,scale=.5]
		{BOOM, A CONTRADICTION!!!};
\end{tikzpicture}}

\renewcommand{\qedsymbol}{$\blacksquare$}

\DeclareMathOperator{\lcm}{lcm}

\newtheorem{theorem}{Theorem}

\newenvironment{exercise}[2][Exercise]{\begin{trivlist}
		\item[\hskip \labelsep {\bfseries #1}\hskip \labelsep {\bfseries #2.}]}{\end{trivlist}}

\setlength\parindent{24pt}

\makeatletter
\renewcommand*\env@matrix[1][*\c@MaxMatrixCols c]{%
	\hskip -\arraycolsep
	\let\@ifnextchar\new@ifnextchar
	\array{#1}}
\makeatother
\setlength\parindent{24pt}


\begin{document}
	
	% --------------------------------------------------------------
	%                         Start here
	% --------------------------------------------------------------
	
	
	\title{Homework 9 (Due Oct 18, 2023)}
	\author{Jack Hyatt\\ %replace with your name
		MATH 554 - Analysis I - Fall 2023} 
	
	\maketitle
	
	Justify all of your answers completely.\\
	
	
	\medskip 
	
	\begin{enumerate}
		\item Prove if $(E,d)$ is a metric space and $F$ is a closed subset of $E$, then $(F,d)$ is also a complete metric space.
		\begin{proof}
			Let $\langle x_n \rangle^\infty_{n=1}$ be a Cauchy sequence in $F$, and consequently in $E$. Since $E$ is complete, we know $p_n$ converges to some $p \in E$. We also know that a subset is closed iff the subset contains the limits of its sequences. So then $p \in F$. This shows $F$ is complete.
		\end{proof}
		
		\item Let $\langle p_n \rangle^\infty_{n=1}$ be a Cauchy sequence in $\R^3$ with its usual metric. Let $p_n = (x_n,y_n,z_n)$.
		\begin{enumerate}
			\item Show that each of the sequence $\langle x_n \rangle^\infty_{n=1}$, $\langle y_n \rangle^\infty_{n=1}$, $\langle z_n \rangle^\infty_{n=1}$ are also Cauchy sequences and explain why this implies the limits $x \coloneq \lim_{n\rightarrow\infty} x_n$, $y \coloneq \lim_{n\rightarrow\infty} y_n$, $z \coloneq \lim_{n\rightarrow\infty} z_n$ exist.
			\begin{proof}
				\[|x_m-x_n| = \sqrt{(x_m-x_n)^2} \leq \sqrt{(x_m-x_n)^2 + (y_m-y_n)^2 + (z_m-z_n)^2}\]\[ = d(p_m,p_n)\]
				Similar calculations can be done for $y$ and $z$.\\
				Since $\langle p_n \rangle^\infty_{n=1}$ is Cauchy, $\exists N>0$ s.t. $m,n>N \implies d(p_m,p_n)<\epsilon$. The above inequalities give us $m,n>N \implies |x_m-x_n| < \epsilon$ and similar ones for $y$ and $z$.\\
				This means the $x$,$y$, and $z$ sequences are Cauchy. Since the individual sequences are in $\R$, which is complete, they will converge as they are Cauchy.\\
			\end{proof}
			
			\item Let $p = (x,y,z)$ and show $\lim_{n\rightarrow\infty} p_n = p$.
			\begin{proof}
				Since the sequences in the part above converge, $\exists N_1>0$, $N_2>0$, and $N_3>0$ s.t.
				\[n>N_1 \implies |x_n-x| < \frac{\epsilon}{\sqrt{3}}\]
				\[n>N_2 \implies |y_n-y| < \frac{\epsilon}{\sqrt{3}}\]
				\[n>N_3 \implies |z_n-z| < \frac{\epsilon}{\sqrt{3}}\]
			Then if $N = \max\{N_1,N_2,N_3\}$ and $p=(x,y,z)$,
			\[n>N \implies ||p_n-p|| = \sqrt{(x_n-x)^2+(y_n-y)^2+(z_n-z)^2}\]
			\[< \sqrt{\left(\frac{\epsilon}{\sqrt{3}}\right)^2 + \left(\frac{\epsilon}{\sqrt{3}}\right)^2 + \left(\frac{\epsilon}{\sqrt{3}}\right)^2} = \epsilon\]
			\end{proof}
			
			\item Conclude that $\R^3$ is a complete metric space.
			\begin{proof}
				The above parts show that an arbitrary Cauchy sequence $\langle p_n \rangle^\infty_{n=1}$ converges, which means that $\R^3$ is complete.
			\end{proof}
		\end{enumerate}
	
		\item Let $\langle p_n \rangle^\infty_{n=1}$ be a Cauchy sequence in the metric space $E$. Prove that the sequence is bounded. That is show there is a ball $B(p,r)$ with $p_n \in B(p,r)$ for all $n$.
		\begin{proof}
			\emph{For this problem, I'm too lazy to put a bar above the ball, but just know all the balls will be closed balls}
			Since $p_n$ is a Cauchy sequence, $\exists N$ s.t.
			\[m,n>N \implies d(p_m,p_n) < \epsilon\]
			So for any $n>m>N$, $p_n \in B(p_m,\epsilon)$.\\
			Let $r = \max\{d(p_m,p_1),d(p_m,p_2),\ldots,d(p_m,p_N),\epsilon\}$.\\
			So then $p_n \in B(p_m,r)$ for all $n$.
		\end{proof}
		
		\item Let $f: E \rightarrow E$ be a contraction and let $\lim_{n\rightarrow\infty} p_n = p$ in $E$. Show $\lim_{n\rightarrow\infty} f(p_n) = f(p)$.
		\begin{proof}
			Want to show $\forall\epsilon>0$, $\exists N$ s.t. $n>N \implies d(f(p_n),f(p))<\epsilon$.\\
			Let $\epsilon > 0$. Since the limit exists in $E$, there is an $N_1$ s.t. $n>N_1 \implies d(p_n,p) < \epsilon$. Let $n>N_1$.\\
			Since $f$ is a contraction, $d(f(p_n),f(p)) \leq \rho d(p_n,p) \leq \rho\epsilon < \epsilon$.\\
			So $d(f(p_n),f(p)) < \epsilon$, which proves the limit.
		\end{proof}
		
		\item Prove the Banach Fixed Point Theorem following the outline given.
		\begin{proof}
			Let $p_0 \in E$ and define a sequence $\langle p_n \rangle^\infty_{n=1}$ where $p_n = f(p_{n-1})$.\\
			Consider $d(p_k,p_{k+1})$ for $k\geq1$.\\
			$d(p_k,p_{k+1}) = d(f(p_{k-1}),f(p_k)) \leq \rho d(p_{k-1},p_k)$. Since we are big boys and girls, we can use our pattern recognition and see that induction will show $d(p_k,p_{k+1}) \leq p^kd(p_0,p_1)$.\\
			Let $m<n$. The triangle implies 
			\[d(p_m,p_n) \leq \sum_{k=m}^{n-1} d(p_k,p_{k+1}) \leq \sum_{k=m}^{n-1} \rho^kd(p_0,p_{1}) = d(p_0,p_1)\sum_{k=m}^{n-1} \rho^k\]
			\[= d(p_0,p_1)\cdot\frac{\rho^m-\rho^n}{1-\rho}\]
			So $d(p_m,p_n) \leq \frac{\rho^m-\rho^n}{1-\rho} d(p_0,p_1) \leq \frac{\rho^m}{1-\rho} d(p_0,p_1)$\\
			Let $m,n \geq N$. Since $0\leq\rho<1$, $\rho^N \geq \rho^m$.\\
			So $d(p_m,p_n) \leq \frac{\rho^m}{1-\rho} d(p_0,p_1) \implies  d(p_m,p_n) \leq \frac{\rho^N}{1-\rho} d(p_0,p_1)$\\
			Let $\epsilon>0$. Since $\lim_{N\rightarrow\infty} \frac{\rho^N}{1-\rho}d(p_0,p_1) = 0$, $\exists N_1$ s.t. $$N>N_1 \implies \frac{\rho^N}{1-\rho}d(p_0,p_1) < \epsilon.$$ So $d(p_m,p_n) < \epsilon$, meaning the sequence $\langle p_n \rangle^\infty_{n=1}$ is Cauchy.\\
			Since $E$ is complete, this means a Cauchy sequence, like $\langle p_n \rangle^\infty_{n=1}$, converges.\\
			Define $p_* \coloneq \lim_{n\rightarrow\infty}p_n$. By problem 4, $\lim_{n\rightarrow\infty}f(p_n) = f(p_*)$. We can say $f(p_n) = p_{n+1}$, and set $n' = n+1$. As $n\rightarrow\infty$, $n'\rightarrow\infty$.\\
			So then $\lim_{n\rightarrow\infty}f(p_n) = f(p_*) 
			\implies \lim_{n'\rightarrow\infty}p_{n'} = f(p_*) \implies
			p_* = f(p_*)$.\\
			So $p_*$ is a fixed point of $f$.\\
			To show that the fixed point is unique, assume that $p_{**}$ is a second fixed point of $f$. Then $d(p_*,p_{**}) = d(f(p_*),f(p_{**})) \leq \rho d(p_*,p_{**})$. This can be repeated an arbitrary amount of times, and by pattern recognition again (since we are big kids) we see that it will approach 0.\\
			So the distance between $p_*$ and $p_{**}$ is 0, meaning they are the same. 
		\end{proof}
		
		\item Let $a \geq 1$ and define $f: [0,\infty) \rightarrow [0,\infty)$ by
		\[f(x) = \sqrt{a+x}\]
		\begin{enumerate}
			\item Show for $x,y \in [0,\infty)$ that 
			\[|f(x) - f(y)| = \frac{|x-y|}{\sqrt{a+x}+\sqrt{a+y}} \leq \frac{|x-y|}{2\sqrt{a}} \leq \frac{1}{2}|x-y|\]
			and therefore $f$ is a contraction. The space $[0,\infty)$ is a complete metric space as it is a closed subset of the complete space $\R$.
			\begin{proof}
				\[|f(x)-f(y)| = |\sqrt{a+x}-\sqrt{a+y}| = \left|\frac{(\sqrt{a+x}-\sqrt{a+y})((\sqrt{a+x}+\sqrt{a+y}))}{(\sqrt{a+x}+\sqrt{a+y})}\right|\]
				\[=\frac{|x-y|}{\sqrt{a+x}+\sqrt{a+y}} \leq \frac{|x-y|}{\sqrt{a}+\sqrt{a}} = \frac{|x-y|}{2\sqrt{a}} \leq \frac{|x-y|}{2}\]
			\end{proof}
			
			\item Define a sequence $x_0 = a$ and $x_{n+1} = f(x_n)$. Find the fixed point.
			\begin{proof}
				Want to find $\lim_{n\rightarrow\infty} x_n$, which can be thought of as finding $\sqrt{a+\sqrt{a+\sqrt{\ldots}}}$.\\
				Since we know the limit exists by the Banach Fixed Point theorem, we can set the limit to $x$. Since $f(x) = x$ (shown in problem 5), we can do
				\[x = \sqrt{a+x} \implies x^2 - x - a = 0 \implies x = \frac{1+\sqrt{1+4a}}{2}\]
			\end{proof}
		\end{enumerate}
		
		\item The Banach Fixed Point Theorem can be used to solve equations that at first glance are not fixed point problems. As an example let us compute numerically a root of the equation
		\[x^3-5x-1=0\]
		We can rewrite this as 
		\[\frac{x^3-1}{5} = x\]
		so we are looking for a fixed point of $f$ given by 
		\[f(x) = \frac{x^3-1}{5} = \frac{x^3}{5} - \frac{1}{5}.\]
		Let $E = [-1,1]$. This is a closed subspace of $\R$ and therefore is a complete metric space.
		\begin{enumerate}
			\item If $|x| \leq 1$ show
			\[|f(x)| \leq \frac{2}{5}\]
			and therefore $f$ maps $E$ into $E$.
			\begin{proof}
				Let $|x|\leq1$. Then $$|f(x)| = \left|\frac{x^3}{5} - \frac{1}{5}\right| \leq \left|\frac{x^3}{5}\right| + \left|\frac{1}{5}\right| = \frac{|x|^3}{5} + \frac{1}{5} \leq \frac{1}{5} + \frac{1}{5} = \frac{2}{5}$$
			\end{proof}
			
			\item Show if $x,y \in E$ then 
			\[|f(x)-f(y)| \leq \frac{3}{5}|x-y|\]
			and therefore $f$ is a contraction on $E = [-1,1]$.
			\begin{proof}
				\[|f(x)-f(y)| = |\frac{x^3-y^3}{5}| = |x-y|\frac{|x^2+xy+y^2|}{5}\]
				\[ \leq |x-y|\frac{|1^2+1\cdot1+1^2|}{5} = \frac{3}{5}|x-y|\]
			\end{proof}
			
			\item This one does not seem like a problem to do, more like an example? So I shan't do it.
		\end{enumerate}
	\end{enumerate}
\end{document}