% !TeX program = lualatex

\documentclass[14pt]{extarticle}



\usepackage[margin=1in]{geometry} 
\usepackage{amsmath,amsthm,amssymb}
\usepackage{MnSymbol}
\usepackage{graphicx}
\usepackage{bm}
\usepackage[normalem,normalbf]{ulem}
\usepackage{algorithm} 
\usepackage{algpseudocode} 
\usepackage{multirow}
\usepackage{rotating}
\usepackage{therefore}

\usepackage{tikz}
\usetikzlibrary{shapes.multipart}
\usetikzlibrary{shapes.symbols}

\usetikzlibrary{graphs,graphdrawing,graphs.standard,quotes}
\usegdlibrary{circular,force,layered,routing}
\tikzset{
	graphs/simpleer/.style={
		nodes={draw,circle, blue, left color=blue!20, text=black, inner sep=1pt},
		node distance=2.5cm, nodes={minimum size=2em}
	},
	every loop/.style={},
}

\newcommand*\circled[1]{\tikz[baseline=(char.base)]{
		\node[shape=circle,draw,inner sep=2pt] (char) {#1};}}

\newcommand{\m}{\medskip\\}
\newcommand{\N}{\mathbb{N}}
\newcommand{\Z}{\mathbb{Z}}
\newcommand{\R}{\mathbb{R}}
\newcommand{\bbs}{\textbackslash\textbackslash\space}
\newcommand{\bs}{\textbackslash\space}
\newcommand{\la}{\enskip\land\enskip}
\newcommand{\lo}{\enskip\lor\enskip}
\newcommand{\comp}[1]{#1^\mathsf{c}}
\newcommand{\micdrop}{\qed}
\newcommand{\contra}{\begin{tikzpicture}
		\node[starburst, draw, minimum width=3cm, minimum height=2cm,line width=1.5pt,red,fill=yellow,scale=.5]
		{BOOM, A CONTRADICTION!!!};
\end{tikzpicture}}

\renewcommand{\qedsymbol}{$\blacksquare$}

\DeclareMathOperator{\lcm}{lcm}

\newtheorem{theorem}{Theorem}

\newenvironment{exercise}[2][Exercise]{\begin{trivlist}
		\item[\hskip \labelsep {\bfseries #1}\hskip \labelsep {\bfseries #2.}]}{\end{trivlist}}

\setlength\parindent{24pt}

\makeatletter
\renewcommand*\env@matrix[1][*\c@MaxMatrixCols c]{%
	\hskip -\arraycolsep
	\let\@ifnextchar\new@ifnextchar
	\array{#1}}
\makeatother
\setlength\parindent{24pt}


\begin{document}
	
	% --------------------------------------------------------------
	%                         Start here
	% --------------------------------------------------------------
	
	
	\title{Homework 15 (Due Nov 27, 2023)}
	\author{Jack Hyatt\\ %replace with your name
		MATH 554 - Analysis I - Fall 2023} 
	
	\maketitle
	
	Justify all of your answers completely.\\
	
	
	\medskip 
	
	\begin{enumerate}
		\item Let $E$ and $E'$ be metric spaces and $f:E\rightarrow E'$ a function. Prove the following are equivalent.
		\begin{enumerate}
			\item $f$ is continuous.
			\item for every $p_0 \in E$, the limit $\lim_{p\rightarrow p_0]} f(p) = f(p_0)$ holds.
			\item  if $\langle p_n \rangle_{n=1}^\infty$ is a sequence in $E$ with $\lim_{n\rightarrow\infty} p_n = p_0$, then $\lim_{n\rightarrow\infty} f(p_n) = f(p_0)$.
			\item If $V$ is an open subset of $E'$, then the preimage $f^{-1}[V]$ is an open subset of $E$.
		\end{enumerate}
		\begin{proof}
			It has been proven before in class that $(a) \iff (b) \iff (c)$, so showing $(a) \iff (d)$ is sufficient.\\
			$(a) \implies (d)$\\
			Assume $f$ is continuous. Let $V\subseteq E'$ be an open set. Let $p_0 \in E$ with $f(p_0) \in V$.\\
			Since $V$ is open, there is $\epsilon>0$ s.t. $B_{E'}(f(p_0),\epsilon)\subseteq V$.\\
			Since $f$ is continuous, $\exists\delta>0$ s.t. $d(p_0,p)<\delta \implies d'(f(p_0),f(p)) < \epsilon$.\\
			So $p \in B_E(p_0,\delta) \implies f(p) \in B_{E'}(f(p_0),\epsilon)\subseteq V$.\\
			So $B_E(p_0,\delta) \subseteq f^{-1}[V]$. This means there is an open ball around every point in $f^{-1}[V]$, making it open.\\
			$(d)\implies(a)$\\
			Assume $(d)$ and let $\epsilon>0$. Let $p_0 \in E$ and let $V=B_{E'}(f(p_0),\epsilon)$.\\
			By $(d)$, $f^{-1}[V]$ contains an open ball around $p_0$, we'll denote $B_E(p_0,\delta)\subseteq f^{-1}[V]$.\\
			So then \[d(p,p_0) < \delta \implies p \in B_E(p_0,\delta) \implies p \in f^{-1}[V] \implies f(p) \in B_{E'}(f(p_0),\epsilon)\]
			\[ \implies d(f(p),f(p_0)) < \epsilon\]
			So $f$ is continuous.
		\end{proof}
	
		\item Let $f : E \rightarrow E'$ be a continuous function between metric spaces and $\U$ an open cover of $E'$. Prove $\{f^{-1}[V] : V \in \U\}$ is an open cover of $E$.
		\begin{proof}
			Assume $f$ is continuous. Let $\U$ be an open cover of $E'$ and towards contradiction, assume $S = \{f^{-1}[V] : V \in \U\}$ is not an open cover of $E$.\\
			By problem (1), we know that $S$ a collection of open sets. So then since it isn't an open cover of $E$, $\exists p \in E$ s.t. $\forall s\in S$, $p \nin s$.\\
			So then $p \nin f^{-1}[V]$ for all $V\in\U$. This means $f(p)\nin V$ for all $V\in\U$.\\
			But $\U$ is an open cover of $E'$ and $f(p)\in E'$. \contra
		\end{proof}
		
		\item What is wrong with the following proof for $\lim_{x\rightarrow1}\frac{1}{x-1} = 0$.
		\begin{proof}
			Let $\epsilon>0$ and set $\delta = |x-1|^2\epsilon$. If $0<|x-1|<\delta$, then
			\begin{align*}
				|\frac{1}{x-1}-0| & = \frac{1}{|x-1|}\\
				&=\frac{1}{|x-1|^2}|x-1|\\
				&<\frac{1}{|x-1|^2}\delta\\
				&=\frac{1}{|x-1|^2}|x-1|^2\epsilon\\
				&=\epsilon
			\end{align*}
			So if $f(x) = 1/(x - 1)$ the inequality $0 < |x - 1| < \delta$ implies $|f(x) - 0| < \epsilon$ which verifies that the definition of $\lim_{x\rightarrow1}f(x) = 0$ holds.
		\end{proof}
		The issue is that we have the definition of $\delta$ depending on $x$. That is a problem since in the definition of limits at points, we define $\delta$ before any $x$.
		
		\item Show that \[\lim_{x\rightarrow1}\frac{1}{x-1}\] does not exist.
		\begin{proof}
			BWOC, assume the limit exists and is equal to $L$.\\
			Since the limit exists, it should hold that $\exists\delta>0$ s.t.
			\[|x-1|<\delta \implies \left|\frac{1}{x-1}-L\right|<1\]
			Let $k = \max(1+L,1+1/\delta)$. Consider $x = 1+1/k$.\\
			We have $|x-1| = |1+1/k-1| = 1/k < \delta$.\\
			We also have $|\frac{1}{x-1}-L| = |\frac{1}{1+1/k-1}-L| = |k-L| \geq k-L \geq 1+L-L = 1 = \epsilon$.
			So $|x-1|<\delta$ and $|f(x)-L|\geq\epsilon$, but we assumed the limit exists, \contra.
		\end{proof}
		
	\end{enumerate}
\end{document}