\documentclass[12pt]{article}

\usepackage[margin=1in]{geometry} 
\usepackage{amsmath,amsthm,amssymb}
\usepackage{graphicx}
\usepackage{bm}

\usepackage{tikz}
\newcommand*\circled[1]{\tikz[baseline=(char.base)]{
		\node[shape=circle,draw,inner sep=2pt] (char) {#1};}}

\newcommand{\N}{\mathbb{N}}
\newcommand{\Z}{\mathbb{Z}}

\newtheorem{theorem}{Theorem}

\newenvironment{exercise}[2][Exercise]{\begin{trivlist}
		\item[\hskip \labelsep {\bfseries #1}\hskip \labelsep {\bfseries #2.}]}{\end{trivlist}}

\makeatletter
\renewcommand*\env@matrix[1][*\c@MaxMatrixCols c]{%
	\hskip -\arraycolsep
	\let\@ifnextchar\new@ifnextchar
	\array{#1}}
\makeatother



\begin{document}
	
	% --------------------------------------------------------------
	%                         Start here
	% --------------------------------------------------------------
	
	
	\title{Homework 6 (Due Oct 3, 2022)}
	\author{Jack Hyatt\\ %replace with your name
		MATH 574 - Discrete Mathamatics - Fall 2022} 
	
	\maketitle
	
	Justify all of your answers completely.\\

\renewcommand{\qedsymbol}{$\blacksquare$}

\begin{enumerate}
\item In this problem, we will prove a one-sided Chebyshev-type bound in several steps. The goal is to fill in the details of the proof of the theorem.

\begin{enumerate}

\item Let $X$ be a random variable and let $c \in \mathbb R$. Prove that $V(X) = V(X+c)$. \\
\begin{proof}
	\[V(X+c) = E((X+c)^2)-E(X+c)^2 = E(X^2+2Xc+c^2) - (E(X)+E(c))^2\]
	\[= E(X^2)+2cE(X)+c^2 - (E(X)^2+2E(X)E(c)+E(c)^2) \]
	\[= E(X^2)+2cE(X)+c^2 - E(X)^2-2cE(X)-c^2 = E(X^2) - E(X)^2 = V(X) \]
\end{proof}

\item Prove that if $Y$ is a random variable with $E(Y) = 0$, then for any constant $c \in \mathbb R$, $E((Y-c)^2) = V(Y) + c^2$. 

\begin{proof}
	\[E((Y-c)^2) = E((Y-c)^2) - E(Y-c)^2 + E(Y-c)^2 =  V(Y-c) + E(Y-c)^2\]
	\[ = V(Y) + (E(Y)-c)^2 = V(Y) + (-c)^2 = V(Y) +c^2\]
	Since $c$ can be any constant, this equality will also be true as $E((Y+c)^2)=V(Y)+c^2$.
\end{proof}

\item Prove that for any random variable $X$, $E(X-E(X)) = 0$. 
\begin{proof}
	\[E(X-E(X)) = E(X)-E(E(X))\]
	Expected value results in a real number, and the expected value of a real number is that same real number, so taking the expected value twice is the same as taking it once. Therefore,
	\[E(X)-E(E(X)) = E(X) - E(X) = 0\]
\end{proof}
\item Prove that for any $c \geq 0$ and random variable $X$, $p(X \geq c) \leq p(X^2 \geq c^2)$.\\\\
\begin{proof}
	$P(X \geq c) \leq P(|X|\geq c)$ since the right side is an upper bound, since the non-negative values of X are unaffected and the negative values become positive and $c$ is non-negative.\\
	$P(|X|\geq c) = P(X^2\geq c^2)$ since both sides of the inequality are positive, they are equivalent.\\ $\therefore P(X \geq c) \leq P(X^2 \geq c^2)$.
\end{proof}

\item Now prove the following theorem. A brief outline is given below, but you should write a full proof. You may cite the results proven above.

\begin{theorem}Let $X$ be a random variable with variance $\sigma^2$. Then  for any $k >0$,
\[p(X-E(X) \geq k) \leq \frac{\sigma^2}{\sigma^2 + k^2}.\]
\end{theorem}

\begin{proof} Set $Y = X - E(X)$.  Then \[p(X-E(X) \geq k)=p(Y \geq k)=p(Y+x \geq k+x)\leq p((Y+x)^2 \geq (k+x)^2),\]

for any $x \geq 0$, using the inequality proved in (d).  

Using Markov's inequality, we obtain 
\[p((Y+x)^2 \geq (k+x)^2) \leq \frac{E((Y+x)^2)}{(k+x)^2}.\] 


Using part (c), we know $E(Y)=0$, and using (b) we conclude that \[\frac{E((Y+x)^2)}{(k+x)^2} = \frac{V(Y)+x^2}{(k+x)^2} = \frac{\sigma^2+x^2}{(k+x)^2}.\]

Therefore $p(X-E(X) \geq k) \leq \frac{\sigma^2+x^2}{(k+x)^2}$, but this holds for {\em any} $x$. Hence this theorem is most useful when we minimize the function $\frac{\sigma^2+x^2}{(k+x)^2}$.

Now we minimize the function $\frac{\sigma^2+x^2}{(k+x)^2}$ with respect to $x$ to get the best bound. The derivative is $\dfrac{2\left(kx-{\sigma}^2\right)}{\left(x+k\right)^3}$ with a zero at $x=\sigma^2/k$. Plugging in that value to the original gives $\frac{\sigma^2+(\frac{\sigma^2}{k})^2}{(k+(\frac{\sigma^2}{k}))^2} = \frac{\sigma^2k^2+\sigma^4}{(k^2+\sigma^2)^2} = \frac{\sigma^2}{\sigma^2 + k^2}$.\end{proof}


\end{enumerate}
\medskip 
\item A biased coin has probability for heads $p= 0.75$. Suppose we flip the coin $1,000$ times. Give an upper bound for the probability that we flip at least $800$ heads
\begin{enumerate}

\item using Markov's inequality.\\\\
Letting $X$ be the number of heads flipped, then $X\sim B(n,p)$, so  $E(X)=750$. So Markov's inequality gives us $p(X \geq 800) \leq \frac{750}{800}$.
\item using Chebyshev's inequality.\\\\
$P(|X-E(X)| \geq k) \leq \frac{V(X)}{k^2} \Leftrightarrow P(|X-750|\geq 50) \leq \frac{187.5}{2500}$
\item using Theorem 1 in the previous question.\\\\
$P(X-750\geq 50) \leq \frac{187.5}{187.5+50^2} = \frac{3}{43}$.
\end{enumerate}


\medskip

\item Let $a_n$ be the number of strings of length $n$ with digits in $0-9$ that contain 2 or more consecutive 0s.
\begin{enumerate}

\item Find a recurrence relation for $a_n$.\\\\
There are 2 cases for a string of length $n$, one where the last digit is 0 and one where it isn't 0. If the last digit is not 0, then there must be double 0's in the $n-1$ substring, which gives a term of $9a_{n-1}$. If the digit is 0, then the second to last digit can also be 0 or not 0. If that second to last digit is 0, this gives us a $10^{n-2}$ term since each digit up to the $n-2$ digit has 10 options. If that second to last digit isn't 0, then this gives us a term of $a_{n-2}$.\\
So our recurrence relation is $a_n = 9a_{n-1} + 9a_{n-2} + 10^{n-2}$. 

\item What are the initial conditions?\\\\
$a_1$ is 0 since consecutive 0's aren't possible, and $a_2$ is 1 since only 00 works. 
\item Determine $a_7$.\\\\
We will have to calculate up to $a_7$ sequentially.\\
$a_3 = 9(1) + 9(0) + 10 = 19$.\\
$a_4 = 9(19) + 9(1) + 10^2 = 280$.\\
$a_5 = 9(280) + 9(19) + 10^3 = 3691$.\\
$a_6 = 9(3691) + 9(280) + 10^4 = 45739$.\\
$a_7 = 9(45739) + 9(3691) + 10^5 = 544870$.\\
\end{enumerate}


\medskip

\item Let $b_n$ be the number of strings of length $n$ with digits in $0-9$ that contain no two repeated numbers in a row.
\begin{enumerate}
\item Determine $b_1$.\\\\
This is trivially 10.
\item Prove that $b_n$ has the recurrence relation $b_n = 9b_{n-1}$ for all $n \geq 2$. \\\\
Let us assume we have a string of $n-1$ digits, we will call the last digit $d$. Adding on a last digit for a string of length $n$, we have 9 options to chose since $d$ will be one of the 10 digits. \qedsymbol
\item Solve the recurrence relation with the initial condition in part (a).\\\\
$b_n = 9b_{n-1} = 9\cdot9b_{n-2} = \ldots = 9\cdot\ldots\cdot9\cdot b_1$, and there will be $n-1$ 9's. $\therefore b_n = 9^{n-1}10$.\qedsymbol
\end{enumerate}

\end{enumerate}
\end{document}
