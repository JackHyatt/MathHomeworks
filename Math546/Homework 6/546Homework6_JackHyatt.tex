% !TeX program = lualatex

\documentclass[12pt]{article}



\usepackage[margin=1in]{geometry} 
\usepackage{amsmath,amsthm,amssymb}
\usepackage{MnSymbol}
\usepackage{graphicx}
\usepackage{bm}
\usepackage[normalem,normalbf]{ulem}
\usepackage{algorithm} 
\usepackage{algpseudocode} 
\usepackage{multirow}
\usepackage{rotating}
\usepackage{therefore}

\usepackage{tikz}
\usetikzlibrary{shapes.multipart}
\usetikzlibrary{shapes.symbols}

\usetikzlibrary{graphs,graphdrawing,graphs.standard,quotes}
\usegdlibrary{circular,force,layered,routing}
\tikzset{
	graphs/simpleer/.style={
		nodes={draw,circle, blue, left color=blue!20, text=black, inner sep=1pt},
		node distance=2.5cm, nodes={minimum size=2em}
	},
	every loop/.style={},
}

\newcommand*\circled[1]{\tikz[baseline=(char.base)]{
		\node[shape=circle,draw,inner sep=2pt] (char) {#1};}}

\newcommand{\m}{\medskip\\}
\newcommand{\N}{\mathbb{N}}
\newcommand{\Z}{\mathbb{Z}}
\newcommand{\R}{\mathbb{R}}
\newcommand{\bbs}{\textbackslash\textbackslash\space}
\newcommand{\bs}{\textbackslash\space}
\newcommand{\la}{\enskip\land\enskip}
\newcommand{\lo}{\enskip\lor\enskip}
\newcommand{\comp}[1]{#1^\mathsf{c}}
\newcommand{\micdrop}{\qed}
\newcommand{\contra}{\begin{tikzpicture}
		\node[starburst, draw, minimum width=3cm, minimum height=2cm,line width=1.5pt,red,fill=yellow,scale=.5]
		{BOOM, A CONTRADICTION!!!};
\end{tikzpicture}}

\renewcommand{\qedsymbol}{$\blacksquare$}

\DeclareMathOperator{\lcm}{lcm}

\newtheorem{theorem}{Theorem}

\newenvironment{exercise}[2][Exercise]{\begin{trivlist}
		\item[\hskip \labelsep {\bfseries #1}\hskip \labelsep {\bfseries #2.}]}{\end{trivlist}}

\setlength\parindent{24pt}

\makeatletter
\renewcommand*\env@matrix[1][*\c@MaxMatrixCols c]{%
	\hskip -\arraycolsep
	\let\@ifnextchar\new@ifnextchar
	\array{#1}}
\makeatother
\setlength\parindent{24pt}


\begin{document}
	
	% --------------------------------------------------------------
	%                         Start here
	% --------------------------------------------------------------
	
	
	\title{Homework 6 (Due Oct 25, 2023)}
	\author{Jack Hyatt\\ %replace with your name
		MATH 546 - Algebraic Structures I - Fall 2023} 
	
	\maketitle
	
	Justify all of your answers completely.\\

	
	\medskip 
	
	\begin{enumerate}
		\item List all the possible orders of elements of the group $G = S_8$. For each possible order, give an example of an element that has that order. Explain why no other orders are possible.\m
		For every $\sigma \in S_8$, we can write it as $\sigma = \tau_1\ldots\tau_k$ for some $k\geq1$, where $\tau_i$ is a cycle.\\
		The possible decomposition types of $\sigma$ will tell us the lengths of each decomposed cycle, and calculating that is the same as calculating partitions. Once we get the decomposition types, we just take the lcm of the partitions.\\
		The possible partitions of 8 and their lcm is:
		\begin{align*}
			1+1+1+1+1+1+1+1 : \lcm(1) = 1 && 2+1+1+1+1+1+1 : \lcm(2,1) = 2\\
			2+2+1+1+1+1 : \lcm(2,1) = 2 && 2+2+2+1+1 : \lcm(2,1) = 2\\
			2+2+2+2 : \lcm(2) = 2 && 3+1+1+1+1+1 : \lcm(3,1) = 3\\
			3+2+1+1+1 : \lcm(3,2,1) = 6 && 3+2+2+1 : \lcm(3,2,1) = 6\\
			3+3+1+1 : \lcm(3,2,1) = 6 && 3+3+2 : \lcm(3,2) = 6\\
			4+1+1+1+1 : \lcm(4,1) = 4 && 4+2+1+1 : \lcm(4,2,1) = 4\\
			4+2+2 : \lcm(4,2) = 4 && 4+3+1 : \lcm(4,3,1) = 12\\
			4+4 : \lcm(4) = 4 && 5+1+1+1 : \lcm(5,1) = 5\\
			5+2+1 : \lcm(5,2,1) = 10 && 5+3 : \lcm(5,3) = 15\\
			6+1+1 : \lcm(6,1) = 6 && 6+2 : \lcm(6,2) = 6\\
			7+1 : \lcm(7,1) = 7 && 8 : \lcm(8) = 8\\
		\end{align*}
		So then the possible orders for $\sigma$ are: 1,2,3,4,5,6,7,8,10,12,15.\\
		An example of an element of order $n$ is just $(1\ 2\ \ldots\ n)$.
		
		\item Let $\tau \in S_n$ be the cycle $(1\,2\,\ldots \, k)$. Prove that for all $\sigma \in S_n$.
		\[\sigma\tau\sigma^{-1} = (\sigma(1)\,\ldots\,\sigma(k))\]
		\begin{proof}
			Let $\tau \in S_n$ be the cycle $(1\,2\,\ldots \, k)$, and $\sigma \in S_n$. Want to show an input of $\sigma(i)$ into $\sigma\tau\sigma^{-1}$ will be consistent with the above cycle.\\
			\textbf{Case 1}: $1\leq i < k$.\\
			Then \[\sigma\tau\sigma^{-1}\sigma(i) = \sigma\tau(i) = \sigma(i+1)\]
			\textbf{Case 2}: $i = k$.\\
			Then \[\sigma\tau\sigma^{-1}\sigma(i) = \sigma\tau(i) = \sigma(1)\]
			\textbf{Case 3}: $k < i \leq n$.\\
			Then \[\sigma\tau\sigma^{-1}\sigma(i) = \sigma\tau(i) = \sigma(i)\]
			So then $\sigma\tau\sigma^{-1}$ is the $(\sigma(1)\,\ldots\,\sigma(k))$ cycle.
		\end{proof}
		
		\item Let $G = S_4$, and let $H = \{\sigma\in S_4 : \sigma^2 = e\}$.
		\begin{enumerate}
			\item List all the elements of $H$.\m
			Obviously $e\in H$.\\
			Since we want the order of $\sigma$ to be at most 2, then the decomposed cycles must be at most length $2$, since the order of a permutation is the lcm of the orders of its cycles. This means the elements can be 2 disjoint cycles of length 2, or a cycle of length 2 with two cycles of length 1.\\
			So $H =$ \{$e$, (1 2)(3 4), (1 3)(2 4), (1 4)(2 3), (1 2), (1 3), (1 4), (2 3), (2 4), (3 4)\}.
			\item Decide whether $H$ is a subgroup of $S_4$ or not.
			\begin{proof}
				It is not due to not being closed, since (1 2)$\circ$(1 3) produces (1 3 2), which is not in $H$.
			\end{proof}
		\end{enumerate}
	
		\item Recall that for a group $G$, $Z(G)$ means the set 
		\[\{x \in G : ax = xa \, \forall a \in G\}.\]
		\begin{enumerate}
			\item Find $Z(G)$ for $G = S_3$.
			\begin{proof}
				$S_3 = $\{e, (1 2 3), (1 3 2), (1 2), (1 3), (2 3)\}\\
				(1 2)$\circ$(2 3) = (1 2 3) and (2 3)$\circ$(1 2) = (1 3 2)\\
				So (1 2),(2 3) $\nin Z(S_3)$.\\
				(1 3)$\circ$(1 3 2) = (2 3) and (1 3 2)$\circ$(1 3) = (1 2)\\
				So (1 3),(1 3 2) $\nin Z(S_3)$.\\
				(1 2 3)$\circ$(1 2) = (1 3) and (1 2)$\circ$(1 2 3) = (2 3)\\
				So (1 2 3) $\nin Z(S_3)$.\\
				Obviously $e$ commutes.\\
				So $Z(S_3) = \{e\}$.
			\end{proof}
			\item If $G = S_4$ and $x = $(1 2 3 4), prove that $x \nin Z(G)$.
			\begin{proof}
				(1 2 3 4)$\circ$(1 2) = (1 3 4) and (1 2)$\circ$(1 2 3 4) = (2 3 4)\\
				So (1 2 3 4) doesn't commute with (1 2), meaning (1 2 3 4) $\nin Z(G)$.
			\end{proof}
		\end{enumerate}
	
		\item Let $G = S_4$. Assume that $H$ is a subgroup of $S_4$ such that every cycle of length 2 in $S_4$ belongs to $H$. Prove that $H$ must be equal to the entire $S_4$.
		\begin{proof}
			Since $H$ contains all cycles of length 2 and $H$ is a subgroup, then $H$ is closed under the composition of those cycles. Since any permutation can be written as the product of cycles of length 2, every permutation in $S_4$ can be written using the length 2 cycles in $H$. Since $H$ is closed, every permutation created using length 2 cycles must also be in $H$. So $H$ contains all of $S_4$.
		\end{proof}
	\end{enumerate}
\end{document}