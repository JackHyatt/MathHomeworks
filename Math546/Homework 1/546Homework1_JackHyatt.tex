% !TeX program = lualatex

\documentclass[12pt]{article}



\usepackage[margin=1in]{geometry} 
\usepackage{amsmath,amsthm,amssymb}
\usepackage{MnSymbol}
\usepackage{graphicx}
\usepackage{bm}
\usepackage[normalem,normalbf]{ulem}
\usepackage{algorithm} 
\usepackage{algpseudocode} 
\usepackage{multirow}
\usepackage{rotating}
\usepackage{therefore}

\usepackage{tikz}
\usetikzlibrary{shapes.multipart}
\usetikzlibrary{shapes.symbols}

\usetikzlibrary{graphs,graphdrawing,graphs.standard,quotes}
\usegdlibrary{circular,force,layered,routing}
\tikzset{
	graphs/simpleer/.style={
		nodes={draw,circle, blue, left color=blue!20, text=black, inner sep=1pt},
		node distance=2.5cm, nodes={minimum size=2em}
	},
	every loop/.style={},
}

\newcommand*\circled[1]{\tikz[baseline=(char.base)]{
		\node[shape=circle,draw,inner sep=2pt] (char) {#1};}}

\newcommand{\m}{\medskip\\}
\newcommand{\N}{\mathbb{N}}
\newcommand{\Z}{\mathbb{Z}}
\newcommand{\R}{\mathbb{R}}
\newcommand{\bbs}{\textbackslash\textbackslash\space}
\newcommand{\bs}{\textbackslash\space}
\newcommand{\la}{\enskip\land\enskip}
\newcommand{\lo}{\enskip\lor\enskip}
\newcommand{\comp}[1]{#1^\mathsf{c}}
\newcommand{\micdrop}{\qed}
\newcommand{\contra}{\begin{tikzpicture}
		\node[starburst, draw, minimum width=3cm, minimum height=2cm,line width=1.5pt,red,fill=yellow,scale=.5]
		{BOOM, A CONTRADICTION!!!};
\end{tikzpicture}}

\renewcommand{\qedsymbol}{$\blacksquare$}

\DeclareMathOperator{\lcm}{lcm}

\newtheorem{theorem}{Theorem}

\newenvironment{exercise}[2][Exercise]{\begin{trivlist}
		\item[\hskip \labelsep {\bfseries #1}\hskip \labelsep {\bfseries #2.}]}{\end{trivlist}}

\setlength\parindent{24pt}

\makeatletter
\renewcommand*\env@matrix[1][*\c@MaxMatrixCols c]{%
	\hskip -\arraycolsep
	\let\@ifnextchar\new@ifnextchar
	\array{#1}}
\makeatother
\setlength\parindent{24pt}


\begin{document}
	
	% --------------------------------------------------------------
	%                         Start here
	% --------------------------------------------------------------
	
	
	\title{Homework 1 (Due Sept 6, 2023)}
	\author{Jack Hyatt\\ %replace with your name
		MATH 546 - Algebraic Structures I - Fall 2023} 
	
	\maketitle
	
	Justify all of your answers completely.\\
	
	
	\medskip 
	
	\begin{enumerate}
		\item Let $m, n$ be positive integers, and let $d = \gcd(m, n)$.
		Consider the following sets:
		\begin{align*}
			S_1 = \{km+ln : k,l\in\Z\}, && S_2=\{dq : q\in\Z\}
		\end{align*}
		Prove that $S_1=S_2$.
		\begin{proof}
			Showing that $S_2\subseteq S_1$.\\
			\[S_2 = \{dq:q\in\Z\} = \{\gcd(n,m)q:q\in\Z\} = \{(km+ln)q:q\in\Z\} \text{ for some integers $k$ and $l$.}\]
			\[= \{(kq)m+(lq)n:q\in\Z\}\]
			$kq$ and $lq$ are both integers, so $\left((kq)m + (lq)n\right) \in S_1$.\\
			Showing that $S_1 \subseteq S_2$.\\
			\[S_1 = \{km+ln : k,l\in\Z\} = \{(d)\left(\frac{km}{d}+\frac{ln}{d}\right) : k,l\in\Z\}\]
			and we know $\frac{km}{d}$ and $\frac{ln}{d}$ are both integers since $d$ is a divisor of $m$ and $n$. So $\left((d)\left(\frac{km}{d}+\frac{ln}{d}\right)\right) \in S_2$. 
		\end{proof}
		\item Recall that we have seen in class (on 8/28) that $f : \Z_5 \rightarrow \Z_{10}, f([x]_5) = [x]_{10}$ is not a well-defined function.
		\begin{enumerate}
			\item Consider $g : \Z_5 \rightarrow \Z_{10}, g([x]_5) = [2x]_{10}$. Is $g$ a well-defined function?
			\begin{proof}
				Let $x_1 \equiv x_2 \pmod{5}$. Then $x_1-x_2 = 5k$ for some integer $k$. Multiplying both sides by $2$ gives, $2x_1 -2x_2 = 10k$. So $2x_1 \equiv 2x_2 \pmod{10}$. \Therefore $g$ is a well-defined function.
			\end{proof}
			\item Consider $h : \Z_5 \rightarrow \Z_{10}, h([x]_5) = [3x]_{10}$. Is $h$ a well-defined function?\m
			No, since $[6]_5 = [1]_5$, but
			$h([6]_5) = [18]_{10} = [1]_{10} \neq [3]_{10} = h([1]_5)$.
		\end{enumerate}
		
		\item \hphantom{word}
		\begin{enumerate}
			\item List all the elements of $\Z^*_{12}$.\m
			$\{[1],[5],[7],[11]\}$
			\item We say that a set $S$ is \textit{closed under addition} if we have $x+y\in S$ for any $x,y \in S$. Is $\Z^*_{12}$ closed under addition?\m
			No, as $[5]+[7] \equiv [0] \nin \Z^*_{12}$.
			\item We say that a set $S$ is \textit{closed under multiplication} if we have $x\cdot y\in S$ for any $x,y \in S$. Is $\Z^*_{12}$ closed under multiplication?\m
			Yes, as $[1]$ times any of the elements is in $\Z^*_{12}$, $[5]\cdot[5] \equiv [1]$, $[5]\cdot[7] \equiv [11]$, $[5]\cdot[11] \equiv [7]$, $[7]\cdot[7] \equiv [1]$, $[7]\cdot[11] \equiv [5]$, $[11]\cdot[11] \equiv [1]$.
		\end{enumerate}
		
		\item Let $p,q$ be prime numbers and let $n$ be a positive integer.
		\begin{enumerate}
			\item Prove that the number of elements of $\Z^*_{p^n}$ is $p^n-p^{n-1}$.
			\begin{proof}
				Let $p,q$ be prime numbers and let $n$ be a positive integer.\\
				Finding the number of integers $0 \leq k \leq p^n-1$ coprime to $p^n$ is equivalent to $|\Z^*_{p^n}|$. Being coprime to a prime power is equivalent to the number not divisible by the prime. In the range $0$ to $p^n-1$, the numbers divisible by $p$ are $0,p,2p,\ldots,(p^{n-1}-2)p,(p^{n-1}-1)p$. So there are $p^{n-1}$ numbers between $0$ and $p^n-1$ divisible by $p$. So there are $p^n-p^{n-1}$ numbers coprime to $p^n$. The rest follows.
			\end{proof}
			\item Assume $p\neq q$. Prove that the number of elements of $\Z^*_{pq}$ is $(p-1)(q-1)$.
			\begin{proof}
				Let $p,q$ be distinct prime numbers.\\
				Finding the number of integers $0 \leq k \leq pq-1$ coprime to $pq$ is equivalent to $|\Z^*_{pq}|$. Since $p$ and $q$ are prime, the only numbers that divide $pq$ are numbers that are divisible by $p$ or $q$.\\
				The numbers in $0$ to $pq-1$ divisible by $p$ are $0,p,2p,\ldots,(q-2)p,(q-1)p$. The amount of those numbers is $q$.\\
				The numbers in $0$ to $pq-1$ divisible by $q$ are $0,q,2q,\ldots,(p-2)q,(p-1)q$. The amount of those numbers is $p$.\\
				The only number in common between the two sets is $0$, so we will have to re-include $0$ by inclusion-exclusion principle.\\
				We now have the number of coprime numbers $pq-p-q+1 = (p-1)(q-1).$
			\end{proof}
		\end{enumerate}
	\end{enumerate}
\end{document}