% !TeX program = lualatex

\documentclass[12pt]{article}



\usepackage[margin=1in]{geometry} 
\usepackage{amsmath,amsthm,amssymb}
\usepackage{MnSymbol}
\usepackage{graphicx}
\usepackage{bm}
\usepackage[normalem,normalbf]{ulem}
\usepackage{algorithm} 
\usepackage{algpseudocode} 
\usepackage{multirow}
\usepackage{rotating}
\usepackage{therefore}

\usepackage{tikz}
\usetikzlibrary{shapes.multipart}
\usetikzlibrary{shapes.symbols}

\usetikzlibrary{graphs,graphdrawing,graphs.standard,quotes}
\usegdlibrary{circular,force,layered,routing}
\tikzset{
	graphs/simpleer/.style={
		nodes={draw,circle, blue, left color=blue!20, text=black, inner sep=1pt},
		node distance=2.5cm, nodes={minimum size=2em}
	},
	every loop/.style={},
}

\newcommand*\circled[1]{\tikz[baseline=(char.base)]{
		\node[shape=circle,draw,inner sep=2pt] (char) {#1};}}

\newcommand{\m}{\medskip\\}
\newcommand{\N}{\mathbb{N}}
\newcommand{\Z}{\mathbb{Z}}
\newcommand{\R}{\mathbb{R}}
\newcommand{\bbs}{\textbackslash\textbackslash\space}
\newcommand{\bs}{\textbackslash\space}
\newcommand{\la}{\enskip\land\enskip}
\newcommand{\lo}{\enskip\lor\enskip}
\newcommand{\comp}[1]{#1^\mathsf{c}}
\newcommand{\micdrop}{\qed}
\newcommand{\contra}{\begin{tikzpicture}
		\node[starburst, draw, minimum width=3cm, minimum height=2cm,line width=1.5pt,red,fill=yellow,scale=.5]
		{BOOM, A CONTRADICTION!!!};
\end{tikzpicture}}

\renewcommand{\qedsymbol}{$\blacksquare$}

\DeclareMathOperator{\lcm}{lcm}

\newtheorem{theorem}{Theorem}

\newenvironment{exercise}[2][Exercise]{\begin{trivlist}
		\item[\hskip \labelsep {\bfseries #1}\hskip \labelsep {\bfseries #2.}]}{\end{trivlist}}

\setlength\parindent{24pt}

\makeatletter
\renewcommand*\env@matrix[1][*\c@MaxMatrixCols c]{%
	\hskip -\arraycolsep
	\let\@ifnextchar\new@ifnextchar
	\array{#1}}
\makeatother
\setlength\parindent{24pt}


\begin{document}
	
	% --------------------------------------------------------------
	%                         Start here
	% --------------------------------------------------------------
	
	
	\title{Homework 10 (Due Dec 8, 2023)}
	\author{Jack Hyatt\\ %replace with your name
		MATH 546 - Algebraic Structures I - Fall 2023} 
	
	\maketitle
	
	Justify all of your answers completely.\\
	
	
	\medskip 
	
	\begin{enumerate}
		\item Find a subgroup of $S_4$ that is isomorphic to $\Z_2\times\Z_2$ by carrying through the procedure we used to prove Cayley's theorem.\m
		Denote elements of $Z_2 \times Z_2$ as $g_1 = ([0],[0]), g_2 = ([1],[0]), g_3 = ([0],[1]), g_4 = ([1],[1])$.\\
		For each $g_i$, associate a $\sigma_i$ by constructing $\sigma_i(\ell) = j$ if $g_i + g_\ell = g_j$.\\
		For $g_1$: $g_1 + g_i = g_i$ for all $i$. So $\sigma_1 = e$\\
		For $g_2$: $g_2 + g_1 = g_2$, $g_2 + g_2 = g_1$, $g_2 + g_3 = g_4$, $g_2 + g_4 = g_3$. So $\sigma_2 = \begin{pmatrix}
			1 & 2 & 3 & 4 \\
			2 & 1 & 4 & 3 \\
		\end{pmatrix} = (1\ 2)(3\ 4)$.\\
		For $g_3$: $g_3 + g_1 = g_3$, $g_3 + g_2 = g_4$, $g_3 + g_3 = g_1$, $g_3 + g_4 = g_2$. So $\sigma_3 = \begin{pmatrix}
			1 & 2 & 3 & 4 \\
			3 & 4 & 1 & 2 \\
		\end{pmatrix} = (1\ 3)(2\ 4)$.\\
		For $g_4$: $g_4 + g_1 = g_4$, $g_4 + g_2 = g_3$, $g_4 + g_3 = g_2$, $g_4 + g_4 = g_1$. So $\sigma_3 = \begin{pmatrix}
			1 & 2 & 3 & 4 \\
			4 & 3 & 2 & 1 \\
		\end{pmatrix} = (1\ 4)(2\ 3)$.\\
		So the subgroup, $H$, of $S_4$ that is isomorphic to $Z_2 \times Z_2$ is\\
		$H = \{e,(1\ 2)(3\ 4),(1\ 3)(2\ 4),(1\ 4)(2\ 4)\}$.
		
		\item Cayley's theorem tells us that there exists a subgroup of $S_6$ that is isomorphic to $Z_6$.
		\begin{enumerate}
			\item  Give an example of such a subgroup and justify the isomorphism.\m
			We know that cyclic groups of the same order are isomorphic. $Z_6$ is a cyclic group of order 6, so we want a cyclic subgroup of $S_6$ also with order 6. The subgroup $\langle(1\ 2\ 3\ 4\ 5\ 6)\rangle$ is a simple example of that.
			
			\item  Does there exist any $n < 6$ such that $\Z_6$ is isomorphic to a subgroup of $S_n$? Find the smallest such value of n.\m
			Since $Z_6$ is cyclic of order 6, the subgroup of $S_n$ it would be isomorphic to would also have to be cyclic of order 6.\\
			So we are looking for values of $n$ that when partitioned, the $\lcm$ of the partitions can be 6.\\
			$n = 5$ works since you can partition 5 into 2+3, so a cyclic subgroup generated from permutation decomposed into cycles of length 2 and 3 would be isomorphic.\\
			$n<5$ cannot be partitioned for the $\lcm$ to equal 6, so 5 is the lowest.
		\end{enumerate}
	
		\item For the group $G$ and the subgroup $H$, list all the cosets with respect to $H$. For each coset, list the elements of the coset. How many distinct cosets are there?
		\begin{enumerate}
			\item $G = S_3$, $H = \{e,(1\ 2)\}$\m
			$G = \{e, (1\ 2), (1\ 3), (2\ 3), (1\ 2\ 3), (1\ 3\ 2)\}$\\
			$eH = \{e,\ (1\ 2)\}$, \quad $(1\ 2)H =\{(1\ 2),\ e\}$, \quad $(1\ 3)H = \{(1\ 3), (1\ 2\ 3)\}$\\ 
			$(2\ 3)H = \{(2\ 3),(1\ 3\ 2)\}$, \quad $(1\ 2\ 3)H = \{(1\ 2\ 3), (1\ 3)\}$\\
			$(1\ 3\ 2)H = \{(1\ 3\ 2), (2\ 3)\}$\\
			So there are 3 distinct cosets.
			\item $G = \Z_4\times\Z_4$, $H=\langle([1]_4,[1]_4)\rangle$.\m
			Since distinct cosets are disjoint, this can cut our search space down a lot.\\
			$e+H = \{([0]_4,[0]_4),([1]_4,[1]_4),([2]_4,[2]_4),([3]_4,[3]_4)\}$\\
			$([1]_4,[0]_4)+H = \{([1]_4,[0]_4),([2]_4,[1]_4),([3]_4,[2]_4),([0]_4,[3]_4)\}$\\
			$([2]_4,[0]_4)+H = \{([2]_4,[0]_4),([3]_4,[1]_4),([0]_4,[2]_4),([1]_4,[3]_4)\}$\\
			$([3]_4,[0]_4)+H = \{([3]_4,[0]_4),([0]_4,[1]_4),([1]_4,[2]_4),([2]_4,[3]_4)\}$\\
			That gives us 4 cosets, each with 4 elements in them, giving 16 total elements. That covers all the elements in $\Z_4\times\Z_4$, so we know no more distinct cosets exists.
		\end{enumerate}
	
		\item For the group $G$ and the subgroup $H$, decide whether $H$ is a normal subgroup of $G$ or not.
		\begin{enumerate}
			\item $G = S_3$, $H = \{e,(1\ 2)\}$\m
			It is not since $(2\ 3) \in G$ and $(1\ 2) \in G$ is a counter example.\\
			The inverse of $(2\ 3)$ is itself. So $(2\ 3)(1\ 2)(2\ 3) = (1\ 3) \nin H$.\\
			So $H$ is not a normal subgroup of $G$.
			\item $G = S_4$, $H=A_4$
			\begin{proof}
				Let $g \in G$ and $h \in H$.\\
				Let us represent $g$ and $h$ as transpositions.\\
				$g = \tau_1\ldots\tau_n$ and $h = \tau_1'\ldots\tau_{2k}'$. $h$ has $2k$ transpositions since $h \in A_4$.\\
				Since the inverse of transpositions is just the order reversed, we get $g^{-1} = \tau_n\ldots\tau_1$.\\
				So then $ghg^{-1} = (\tau_1\ldots\tau_n)(\tau_1'\ldots\tau_{2k}')(\tau_n\ldots\tau_1)$.\\
				$ghg^{-1}$ has $n+2k+n = 2(n+k)$ transpositions, which is an even amount.\\
				So $ghg^{-1} \in H$, making $H$ a normal subgroup.
			\end{proof}
		\end{enumerate}
		
		\item Let $G=\Z_4\times\Z_6$, and let $H=\langle([1]_4,[0]_6)\rangle$. Consider the factor group $G/H$.
		\begin{enumerate}
			\item What is the order of the element $([1]_4,[2]_6)$ as an element of $G$?\m
			$[1]_4$ has order 4 and $[2]_6$ has order 3, so the order of $([1]_4,[2]_6)$ will be the $\lcm(3,4)$, which is 12.
			\item What is the order of the element $([1]_4,[2]_6)+H$ as an element of $G/H$?\m
			To find the order of the element, want to find the smallest $n$ s.t. $n\cdot(([1]_4,[2]_6)+H)$ is the identity of $G/H$, which is $e+H = H$.\\
			$(([1]_4,[2]_6)+H) \neq H$.\\
			$(([1]_4,[2]_6)+H)+(([1]_4,[2]_6)+H) = (([2]_4,[4]_6)+H) \neq H$.\\
			$(([1]_4,[2]_6)+H)+(([1]_4,[2]_6)+H)+(([1]_4,[2]_6)+H) = (([2]_4,[4]_6)+H)+(([1]_4,[2]_6)+H) = (([3]_4,[0]_6)+H) = H$ since $([3]_4,[0]_6)\in H$.\\
			So the order is 3.
			
		\end{enumerate}
			
			
	\end{enumerate}
\end{document}