% --------------------------------------------------------------
% This is all preamble stuff that you don't have to worry about.
% Head down to where it says "Start here"
% --------------------------------------------------------------
 
\documentclass[12pt]{article}

\usepackage[margin=1in]{geometry} 
\usepackage{amsmath,amsthm,amssymb}
\usepackage{graphicx}
\usepackage{bm}

\usepackage{tikz}
\newcommand*\circled[1]{\tikz[baseline=(char.base)]{
		\node[shape=circle,draw,inner sep=2pt] (char) {#1};}}
	
\newcommand{\N}{\mathbb{N}}
\newcommand{\Z}{\mathbb{Z}}

\newenvironment{exercise}[2][Exercise]{\begin{trivlist}
		\item[\hskip \labelsep {\bfseries #1}\hskip \labelsep {\bfseries #2.}]}{\end{trivlist}}

\makeatletter
\renewcommand*\env@matrix[1][*\c@MaxMatrixCols c]{%
	\hskip -\arraycolsep
	\let\@ifnextchar\new@ifnextchar
	\array{#1}}
\makeatother
 
\begin{document}
 
% --------------------------------------------------------------
%                         Start here
% --------------------------------------------------------------
 
%\renewcommand{\qedsymbol}{\filledbox}
 
\title{Homework 4 (Due Sep 23, 2022)}
\author{Jack Hyatt\\ %replace with your name
MATH 570 - Discrete Optimization - Fall 2022} %if necessary, replace with your course title

\maketitle

Justify all of your answers completely.\\

Note: Going from one tableau to the next, the pivot element will be circled.

\begin{exercise}{1} Solve the following LP using the simplex algorithm with tableaux.
\begin{equation*}
\begin{array}{ll@{}ll}
\text{minimize}  & \displaystyle 2x_1+3x_2 &\\
\text{subject to}& \displaystyle x_1+x_2 \leq 10   \\
                 & \displaystyle x_1+2x_2 \geq 12 \\
                 & \displaystyle 2x_1+x_2 \geq 12 \\
                 & \displaystyle 0 \leq x_1,x_2
\end{array}
\end{equation*}
This is not origin feasible, so we must rewriting in standard form and use an auxiliary tableau.
\begin{equation*}
	\begin{array}{ll@{}ll}
		\text{maximize}  & \displaystyle -x_0 &\\
		\text{original obj}  & \displaystyle -2x_1-3x_2 &\\
		\text{subject to}& \displaystyle -x_0+x_1+x_2 \leq 10   \\
		& \displaystyle -x_0-x_1-2x_2 \leq -12 \\
		& \displaystyle -x_0-2x_1-x_2 \leq -12 \\
		& \displaystyle 0 \leq x_0,x_1,x_2
	\end{array}
\end{equation*}
The tableau looks like
\[\begin{bmatrix}[cccccc|c]
	-1 &  1 &  1 & 1 & 0 & 0 &  10\\
	\circled{-1} & -1 & -2 & 0 & 1 & 0 & -12\\
	-1 & -2 & -1 & 0 & 0 & 1 & -12\\
	 0 & -2 & -3 & 0 & 0 & 0 &  0\\
	-1 &  0 &  0 & 0 & 0 & 0 &  0\\
\end{bmatrix}\]
\[\begin{bmatrix}[cccccc|c]
	 0 &  \circled{2} &  3 & 1 & -1 & 0 & 22\\
	 1 &  1 &  2 & 0 & -1 & 0 & 12\\
	 0 & -1 &  1 & 0 & -1 & 1 &  0\\
	 0 & -2 & -3 & 0 &  0 & 0 &  0\\
	 0 &  1 &  2 & 0 & -1 & 0 &  12\\
\end{bmatrix}\]
\[\begin{bmatrix}[cccccc|c]
	0 & 1 & \frac{3}{2} & \frac{1}{2} & -\frac{1}{2} & 0 & 11\\
	1 & 0 & \circled{1/2} &-\frac{1}{2} & -\frac{1}{2} & 0 &  1\\
	0 & 0 & \frac{5}{2} & \frac{1}{2} & -\frac{3}{2} & 1 & 11\\
	0 & 0 &  	0		&	  1 	  & 	  -1	 & 0 & 22\\
	0 & 0 & \frac{1}{2} &-\frac{1}{2} & -\frac{1}{2} & 0 &  1\\
\end{bmatrix}\]
\[\begin{bmatrix}[cccccc|c]
	-3 & 1 & 0 & 2 & 1 & 0 & 8\\
	2 & 0 & 1 & -1 & -1 & 0 &  2\\
	-5 & 0 & 0 & 6 & 1 & 1 & 6\\
	0 & 0 &  0 &  1 &  -1 & 0 & 22\\
	-1 & 0 & 0 & 0 & 0 & 0 &  0\\
\end{bmatrix}\]
Now to extract the normal tableau from this aux one. It will be 
\[\begin{bmatrix}[ccccc|c]
	 1 & 0 & \circled{2} & 1 & 0 & 8\\
	 0 & 1 & -1 & -1 & 0 &  2\\
	 0 & 0 & 6 & 1 & 1 & 6\\
	 0 &  0 &  1 &  -1 & 0 & 22\\
\end{bmatrix}\]
\[\begin{bmatrix}[ccccc|c]
	\frac{1}{2}  & 0 &  1 &  \frac{1}{2} & 0 &  4\\
 	\frac{1}{2}	 & 1 &  0 & -\frac{1}{2} & 0 &  6\\
		-3		 & 0 &  0 & 	-2		 & 1 & 18\\
	-\frac{1}{2} & 0 &  0 & -\frac{3}{2} & 0 & 18\\
\end{bmatrix}\]
This is the optimal tableau and tells us that the optimal value of the original LP is 18 at the point (0,6).
\end{exercise}

\begin{exercise}{2} Solve the following LP and its dual using the simplex algorithm.
\begin{equation*}
\begin{array}{ll@{}ll}
\text{maximize}  & \displaystyle 3x_1+2x_2-5x_3 &\\
\text{subject to}& \displaystyle 4x_1-2x_2+2x_3 \leq 4   \\
                 & \displaystyle 2x_1-x_2+x_3 \leq 1 \\
                 & \displaystyle 0 \leq x_1,x_2,x_3
\end{array}
\end{equation*}
The LP in its tableau is 
\[\begin{bmatrix}[ccccc|c]
	4 & -2 & 2 & 1 & 0 & 4\\
	2 & -1 & 1 & 0 & 1 & 1\\
	3 & 2 & -5 & 0 & 0 & 0
\end{bmatrix}\]
Since we can see that the column for $x_2$ has a positive entry in the Z row and negative or zeros everywhere else, this means that the objective function can be increased arbitrarily by $x_2$ while still satisfying the constraints. This means the LP is unbounded, which also makes the dual infeasible by weak duality theorem. So the value for $\boldsymbol{P}$ is $\infty$ and $\boldsymbol{D}$ is $-\infty$.

\end{exercise}

\begin{exercise}{3} Solve the following LP using the simplex algorithm.
\begin{equation*}
\begin{array}{ll@{}ll}
\text{minimize}  & \displaystyle 6x_1+3x_2 &\\
\text{subject to}& \displaystyle x_1 + x_2 \geq 1   \\
                 & \displaystyle 2x_1 - x_2 \geq 1 \\
                 & \displaystyle 3x_2 \leq 2 \\
                 & \displaystyle 0 \leq x_1,x_2
\end{array}
\end{equation*}
This is not origin feasible, so we must rewriting in standard form and use an auxiliary tableau.
\begin{equation*}
	\begin{array}{ll@{}ll}
		\text{maximize}  & \displaystyle -x_0 &\\
		\text{original obj}  & \displaystyle -6x_1-3x_2 &\\
		\text{subject to}& \displaystyle -x_0 -x_1 - x_2 \leq -1   \\
		& \displaystyle -x_0 -2x_1 + x_2 \leq -1 \\
		& \displaystyle -x_0 + 0x_1+ 3x_2 \leq 2 \\
		& \displaystyle 0 \leq x_0,x_1,x_2
	\end{array}
\end{equation*}
The tableau looks like
\[\begin{bmatrix}[cccccc|c]
	\circled{-1} & -1 & -1 & 1 & 0 & 0 & -1\\
	-1 & -2 &  1 & 0 & 1 & 0 & -1\\
	-1 &  0 &  3 & 0 & 0 & 1 &  2\\
	 0 & -6 & -3 & 0 & 0 & 0 &  0\\
	-1 &  0 &  0 & 0 & 0 & 0 &  0\\
\end{bmatrix}\]
\[\begin{bmatrix}[cccccc|c]
	1 &  \circled{1} &  1 & -1 & 0 & 0 & 1\\
	0 & -1 &  2 & -1 & 1 & 0 & 0\\
	0 &  1 &  4 & -1 & 0 & 1 & 3\\
	0 & -6 & -3 &  0 & 0 & 0 &  0\\
	0 &  1 &  1 & -1 & 0 & 0 &  1\\
\end{bmatrix}\]
\[\begin{bmatrix}[cccccc|c]
	1 &  1 &  1 & -1 & 0 & 0 & 1\\
	0 &  0 &  3 & -2 & 1 & 0 & 1\\
	0 &  0 &  3 & -2 & 0 & 1 & 2\\
	0 &  0 &  3 & -6 & 0 & 0 & 6\\
	0 &  0 &  0 &  0 & 0 & 0 & 0\\
\end{bmatrix}\]
Since the bottom right element got to 0, P is feasible and can do normal simplex on the tableau ignoring the bottom row and left column.
\[\begin{bmatrix}[ccccc|c]
	1 &  1 & -1 & 0 & 0 & 1\\
	0 &  \circled{3} & -2 & 1 & 0 & 1\\
	0 &  3 & -2 & 0 & 1 & 2\\
	0 &  3 & -6 & 0 & 0 & 6\\
\end{bmatrix}\]
\[\begin{bmatrix}[ccccc|c]
	1 &  0 & -\frac{1}{3} & -\frac{1}{3} & 0 & \frac{2}{3}\\
	0 &  1 & -\frac{2}{3} & \frac{1}{3} & 0 & \frac{1}{3}\\
	0 &  0 &  0 & -1 & 1 & 1\\
	0 &  0 & -4 & -1 & 0 & 5\\
\end{bmatrix}\]
Simplex has ended with the minimum value being -5, at the point $(\frac{2}{3},\frac{1}{3})$.
\end{exercise}

\begin{exercise}{4} In Section 3.1 (page 5), we see an example due to Kuhn of cycling because a poor choice of pivots is made.  (This is Dantzig's original rule: choose the lowest-cost variable to enter, i.e., the largest value in the objective/cost row, and minimize indices when there is any kind of tie.)  Show the cycle that occurs, and give the list of corresponding tableaux, bases (i.e., the list of basic variables), and degenerate pivots.  Then show the solution using Bland's rule instead.
\begin{equation*}
\begin{array}{ll@{}ll}
\text{maximize}  & \displaystyle 2x_1+3x_2-x_3-12x_4 &\\
\text{subject to}& \displaystyle -2x_1-9x_2+x_3+9x_4 \leq 0   \\
                 & \displaystyle \frac{1}{3}x_1+x_2-\frac{1}{3}x_3-2x_4 \leq 0 \\
                 & \displaystyle 2x_1+3x_2-x_3-12x_4 \leq 2 \\
                 & \displaystyle 0 \leq x_1,x_2,x_3,x_4
\end{array}
\end{equation*}
The LP as a tableau is, and we will use Dantzig's rule
\[\begin{bmatrix}[ccccccc|c]
	    -2      & -9 &     1        &   9 & 1 & 0 &  0 & 0\\
	\frac{1}{3} &  \circled{1} & -\frac{1}{3} &  -2 & 0 & 1 &  0 & 0\\
	     2      &  3 &    -1        & -12 & 0 & 0 &  1 & 2\\
	     2      &  3 &    -1        & -12 & 0 & 0 &  0 & 0
\end{bmatrix}\]
$x_2$ is entering the basis and $s_2$ (slack var) is leaving.
\[\begin{bmatrix}[ccccccc|c]
	\circled{1} &  0 &     -2       &  -9 & 1 &  9 &  0 & 0\\
	\frac{1}{3} &  1 & -\frac{1}{3} &  -2 & 0 &  1 &  0 & 0\\
	 	 1      &  0 &     0        &  -6 & 0 & -3 &  1 & 2\\
		 1      &  0 &     0        &  -6 & 0 & -3 &  0 & 0
\end{bmatrix}\]
$x_1$ is entering the basis and $s_1$ is leaving.
\[\begin{bmatrix}[ccccccc|c]
		 1 &  0 &     -2       &  -9 &  1            &   9 &  0 & 0\\
		 0 &  1 & \frac{1}{3}  &   \circled{1} &  -\frac{1}{3} &  -2 &  0 & 0\\
		 0 &  0 &     2        &   3 & -1			  & -12 &  1 & 2\\
	 	 0 &  0 &     2        &   3 & -1 			  & -12 &  0 & 0
\end{bmatrix}\]
$x_4$ is entering the basis and $x_2$ is leaving.
\[\begin{bmatrix}[ccccccc|c]
	 1 &  9 & \circled{1}  &  0 &  -2           &  -9 &  0 & 0\\
	 0 &  1 & \frac{1}{3}  &  1 &  -\frac{1}{3} &  -2 &  0 & 0\\
	 0 & -3 &     1        &  0 &  0			 &  -6 &  1 & 2\\
	 0 & -3 &     1        &  0 &  0 			 &  -6 &  0 & 0
\end{bmatrix}\]
$x_3$ is entering the basis and $x_1$ is leaving.
\[\begin{bmatrix}[ccccccc|c]
	 1 			  &   9 & 1 &  0 &  -2          &  -9 &  0 & 0\\
	 -\frac{1}{3} &  -2 & 0 &  1 &  \frac{1}{3} &   \circled{1} &  0 & 0\\
	-1 			  & -12 & 0 &  0 &  2		    &   3 &  1 & 2\\
	-1 			  & -12 & 0 &  0 &  2 			&   3 &  0 & 0
\end{bmatrix}\]
$s_2$ is entering the basis and $x_4$ is leaving.
\[\begin{bmatrix}[ccccccc|c]
	-2 			  & -9 & 1 &  9 &  \circled{3} &  0 &  0 & 0\\
	-\frac{1}{3}  & -2 & 0 &  1 &  \frac{1}{3} &  1 &  0 & 0\\
	0 			  & -6 & 0 & -3 &  1		   &  0 &  1 & 2\\
	0 			  & -6 & 0 & -3 &  1 		   &  0 &  0 & 0
\end{bmatrix}\]
$s_1$ is entering the basis and $x_3$ is leaving.
\[\begin{bmatrix}[ccccccc|c]
	-\frac{2}{3} & -3 & \frac{1}{3} &  3 &  1 &  0 &  0 & 0\\
	-\frac{1}{9} & -1 & \frac{1}{9} &  0 & 0 &  1 &  0 & 0\\
	\frac{2}{3}  & -3 & -\frac{1}{3}& -6 &  0 &  0 &  1 & 2\\
	\frac{2}{3}  & -3 & -\frac{1}{3}& -6 &  0 &  0 &  0 & 0
\end{bmatrix}\]
I'm sure there are some arithmetic errors going from one tableau to the next, but regardless, we are back to a basis of just $s_{1,2,3}$. Since we have reached the same dictionary as we begun with, we know we will cycle. Now to solve the LP using Bland's rule.
\[\begin{bmatrix}[ccccccc|c]
	-2   	      & -9 &     1        &   9 & 1 & 0 &  0 & 0\\
	\circled{1/3} &  1 & -\frac{1}{3} &  -2 & 0 & 1 &  0 & 0\\
	2     		  &  3 &    -1        & -12 & 0 & 0 &  1 & 2\\
	2      		  &  3 &    -1        & -12 & 0 & 0 &  0 & 0
\end{bmatrix}\]
\[\begin{bmatrix}[ccccccc|c]
	0 & -3 & -1 & -3 & 1 &  6 &  0 & 0\\
	1 &  3 & -1 & -6 & 0 &  3 &  0 & 0\\
	0 & -3 &  \circled{1} &  0 & 0 & -6 &  1 & 2\\
	0 & -3 &  1 &  0 & 0 & -6 &  0 & 0
\end{bmatrix}\]
\[\begin{bmatrix}[ccccccc|c]
	0 & -6 & 0 & -3 & 1 &  0 &  1 & 2\\
	1 &  0 & 0 & -6 & 0 & -3 &  1 & 2\\
	0 & -3 & 1 &  0 & 0 & -6 &  1 & 2\\
	0 &  0 & 0 &  0 & 0 &  0 & -1 & -2
\end{bmatrix}\]
This is the optimal tableau with the optimal value of 2 at point (2,0,2,0). This logically also makes sense since the objective function was also the same as a constraint, constrained below or equal to 2.
\end{exercise}
 
\end{document}

