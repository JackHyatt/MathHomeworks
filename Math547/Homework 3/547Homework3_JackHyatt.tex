% !TeX program = lualatex

\documentclass[12pt]{article}



\usepackage[margin=1in]{geometry} 
\usepackage{amsmath,amsthm,amssymb}
\usepackage{MnSymbol}
\usepackage{graphicx}
\usepackage{bm}
\usepackage[normalem,normalbf]{ulem}
\usepackage{algorithm} 
\usepackage{algpseudocode} 
\usepackage{multirow}
\usepackage{rotating}
\usepackage{therefore}

\usepackage{tikz}
\usetikzlibrary{shapes.multipart}
\usetikzlibrary{shapes.symbols}

\usetikzlibrary{graphs,graphdrawing,graphs.standard,quotes}
\usegdlibrary{circular,force,layered,routing}
\tikzset{
	graphs/simpleer/.style={
		nodes={draw,circle, blue, left color=blue!20, text=black, inner sep=1pt},
		node distance=2.5cm, nodes={minimum size=2em}
	},
	every loop/.style={},
}

\newcommand*\circled[1]{\tikz[baseline=(char.base)]{
		\node[shape=circle,draw,inner sep=2pt] (char) {#1};}}

\newcommand{\m}{\medskip\\}
\newcommand{\N}{\mathbb{N}}
\newcommand{\Z}{\mathbb{Z}}
\newcommand{\R}{\mathbb{R}}
\newcommand{\bbs}{\textbackslash\textbackslash\space}
\newcommand{\bs}{\textbackslash\space}
\newcommand{\la}{\enskip\land\enskip}
\newcommand{\lo}{\enskip\lor\enskip}
\newcommand{\comp}[1]{#1^\mathsf{c}}
\newcommand{\micdrop}{\qed}
\newcommand{\contra}{\begin{tikzpicture}
		\node[starburst, draw, minimum width=3cm, minimum height=2cm,line width=1.5pt,red,fill=yellow,scale=.5]
		{BOOM, A CONTRADICTION!!!};
\end{tikzpicture}}

\renewcommand{\qedsymbol}{$\blacksquare$}

\DeclareMathOperator{\lcm}{lcm}

\newtheorem{theorem}{Theorem}

\newenvironment{exercise}[2][Exercise]{\begin{trivlist}
		\item[\hskip \labelsep {\bfseries #1}\hskip \labelsep {\bfseries #2.}]}{\end{trivlist}}

\setlength\parindent{24pt}

\makeatletter
\renewcommand*\env@matrix[1][*\c@MaxMatrixCols c]{%
	\hskip -\arraycolsep
	\let\@ifnextchar\new@ifnextchar
	\array{#1}}
\makeatother
\setlength\parindent{24pt}


\begin{document}
	
	% --------------------------------------------------------------
	%                         Start here
	% --------------------------------------------------------------
	
	
	\title{Homework 3 (Due Feb 7, 2025)}
	\author{Jack Hyatt\\ %replace with your name
		MATH 547 - Algebraic Structures II - Spring 2025} 
	
	\maketitle
	
	Justify all of your answers completely.\\
	
	
	\medskip 
	
	\begin{enumerate}
		\item Use a substitution and Eisenstein's criterion to prove that each of the polynomials below is irreducible over $\Q$.
		\begin{enumerate}
			\item $f(x) = x^6 + x^3 + 1$\m
			Let us substitute with x+1.
			\begin{align*}
				f(x+1) &= (x+1)^6 + (x+1)^3 + 1\\
				&= x^6 + 6x^5 + 15x^4 + 21x^3 + 18x^2 + 9x + 3
			\end{align*}
			Let our prime be $p=3$. We have that $p \nmid 1$, $p$ divides all other coefficients, and $p^2 \nmid 3$.
			
			So $f(x)$ must be irreducible over $\Q[x]$.

			\item $f(x) = x^3 + 3x^2 + 5x + 5$\m
			\begin{align*}
				f(x+1) &= (x+1)^3 + 3(x+1)^2 + 5(x+1) + 5\\
				&= x^3 + 6x^2 + 14x + 14
			\end{align*}
			Let our prime be $p=2$. We have that $p \nmid 1$, $p$ divides all other coefficients, and $p^2 \nmid 14$.
			
			So $f(x)$ must be irreducible over $\Q[x]$.
		\end{enumerate}
	
		\item Find the factorization of $f(x) = x^8 - 1$ into irreducible factors in $\Q[x]$. Prove that the factors you found are irreducible in $\Q[x]$.\m
		\[f(x) = x^8-1 = (x^4+1)(x^4-1) = (x^4+1)(x^2+1)(x+1)(x-1)\]
		Clearly $(x^2+1)$ is irreducible since it is of degree $\leq 3$ with no roots in $\Q$, and clearly the linear terms $(x-1)$ and $(x+1)$ are irreducible. Now to show irreduciblity for $(x^4+1)$.
		
		Consider substitution with $x+1$.
		\[f(x+1) = x^4 + 4x^3 + 6x^2 + 4x + 2\]
		Clearly, with the prime being 2, Eisenstein's criterion holds. So We have all irreducible factors.
		
		\item Find the factorization of $f(x) = x^{12} - 1$ into irreducible factors in $\Q[x]$. Prove that the factors you found are irreducible in $\Q[x]$.\m
		\begin{align*}
			f(x) = x^{12}-1 &= (x^6+1)(x^6-1)\\
			&= (x^2+1)(x^4-x^2+1)(x^3+1)(x^3-1)\\
			&= (x^2+1)(x^4-x^2+1)(x+1)(x^2-x+1)(x-1)(x^2+x+1)
		\end{align*}
		Clearly $(x^2+1)$,$(x^2-x+1)$, and $(x^2+x+1)$ are irreducible since they are degree $\leq 3$ with no roots in $\Q$, and clearly the linear terms $(x-1)$ and $(x+1)$ are irreducible. Now to show irreduciblity for $(x^4-x^2+1)$.
		
		Let us first show it cannot factor into two quadratic factors. Let $y = x^2$. Then $x^4-x^2+1$ becomes $y^2-y+1$, which has no real solutions. So no quadratic factors.\\
		Now to show it cannot factor with a linear factor. By rational root theorem, the only possible factors could be $x\pm1$. But $(\pm1)^4-(\pm1)^2+1 = 1 \neq 0$. So no linear factors.\\
		So all the factors found are irreducible.
		
		
		\item Let $m$ and $n$ be positive integers. Prove that $(x^m - 1) \divides (x^n-1)$ in $\Q[x]$ if and only if $m \divides n$. Hint: It might be useful to look at the complex roots of these polynomials.
		\begin{proof}
			($\implies$)\\
			Since $(x^m - 1) \divides (x^n-1)$, then every root, $r$, of $x^m - 1$ is also a root of $x^n-1$. Since $r$ is a $m$'th root of unity, we shall represent the root as
			\[r = \cos(\frac{2\pi}{m}k) + i\sin(\frac{2\pi}{m}k)\]
			for some $k$. Since $r$ is also an $n$'th root of unity, we have
			\[1 = r^n = \cos(2\pi k\frac{n}{m}) + i\sin(2\pi k\frac{n}{m}).\]
			So then $n/m$ is an integer, meaning $m \divides n$.
			
			($\impliedby$)\\
			Let $n = km$ for some integer $k$. Then
			\[x^n-1 = x^{km}-1 = (x^m)^k - 1^k = (x^m-1)(x^{(k-1)m}+x^{(k-2)m}+\ldots+x^m+1)\]
			So $(x^m - 1) \divides (x^n-1)$ in $\Q[x]$.
		\end{proof}
		
	\end{enumerate}
\end{document}