% !TeX program = lualatex

\documentclass[12pt]{article}



\usepackage[margin=1in]{geometry} 
\usepackage{amsmath,amsthm,amssymb}
\usepackage{MnSymbol}
\usepackage{graphicx}
\usepackage{bm}
\usepackage[normalem,normalbf]{ulem}
\usepackage{algorithm} 
\usepackage{algpseudocode} 
\usepackage{multirow}
\usepackage{rotating}
\usepackage{therefore}

\usepackage{tikz}
\usetikzlibrary{shapes.multipart}
\usetikzlibrary{shapes.symbols}

\usetikzlibrary{graphs,graphdrawing,graphs.standard,quotes}
\usegdlibrary{circular,force,layered,routing}
\tikzset{
	graphs/simpleer/.style={
		nodes={draw,circle, blue, left color=blue!20, text=black, inner sep=1pt},
		node distance=2.5cm, nodes={minimum size=2em}
	},
	every loop/.style={},
}

\newcommand*\circled[1]{\tikz[baseline=(char.base)]{
		\node[shape=circle,draw,inner sep=2pt] (char) {#1};}}

\newcommand{\m}{\medskip\\}
\newcommand{\N}{\mathbb{N}}
\newcommand{\Z}{\mathbb{Z}}
\newcommand{\R}{\mathbb{R}}
\newcommand{\bbs}{\textbackslash\textbackslash\space}
\newcommand{\bs}{\textbackslash\space}
\newcommand{\la}{\enskip\land\enskip}
\newcommand{\lo}{\enskip\lor\enskip}
\newcommand{\comp}[1]{#1^\mathsf{c}}
\newcommand{\micdrop}{\qed}
\newcommand{\contra}{\begin{tikzpicture}
		\node[starburst, draw, minimum width=3cm, minimum height=2cm,line width=1.5pt,red,fill=yellow,scale=.5]
		{BOOM, A CONTRADICTION!!!};
\end{tikzpicture}}

\renewcommand{\qedsymbol}{$\blacksquare$}

\DeclareMathOperator{\lcm}{lcm}

\newtheorem{theorem}{Theorem}

\newenvironment{exercise}[2][Exercise]{\begin{trivlist}
		\item[\hskip \labelsep {\bfseries #1}\hskip \labelsep {\bfseries #2.}]}{\end{trivlist}}

\setlength\parindent{24pt}

\makeatletter
\renewcommand*\env@matrix[1][*\c@MaxMatrixCols c]{%
	\hskip -\arraycolsep
	\let\@ifnextchar\new@ifnextchar
	\array{#1}}
\makeatother
\setlength\parindent{24pt}


\begin{document}
	
	% --------------------------------------------------------------
	%                         Start here
	% --------------------------------------------------------------
	
	
	\title{Homework 1 (Due Jan 22, 2025)}
	\author{Jack Hyatt\\ %replace with your name
		MATH 547 - Algebraic Structures II - Spring 2025} 
	
	\maketitle
	
	Justify all of your answers completely.\\
	
	
	\medskip 
	
	\begin{enumerate}
		\item Let $R$ be a ring and let $S = R[[X]]$ be the ring of formal power series with coefficients in $R$. Let $f = \sum_{n=0}^{\infty} a_n X^n$ be an element in $S$, where $a_n \in R$ for all $n \geq 0$.
		Prove that $f$ is a unit in $S$ if and only if $a_0$ is a unit in $R$.
		
		\begin{proof}
			$(\implies)$\\
			Assume $f$ is a unit in $S$. Then let us denote $f^{-1} = \sum_{n=0}^{\infty} a'_n X^n$ as the multiplicative inverse of $f$.
			\[1 = ff^{-1} = \sum_{n=0}^{\infty}(\sum_{k=0}^{n} a_ka'_{n-k})X^n\]
			So then $\sum_{k=0}^{n} a_ka'_{n-k} = 0$ for every $n\neq0$ and $a_0a'_0 = 1$. A similar argument can be made for $a'_0a_0=1$, meaning that $a'_0$ is the inverse of $a_0$, making $a_0$ a unit in $R$.
			
			$(\impliedby)$\\
			Let $f = \sum_{n=0}^{\infty} a_n X^n$ be an element in $S$, where $a_n \in R$ for all $n \geq 0$. Assume $a_0$ is a unit in $R$. Then let us denote $a_0^{-1}$ as the multiplicative inverse of $a_0$.
			
			Now to construct $f^{-1} = \sum_{n=0}^{\infty} a'_n X^n$. We would need $1 = ff^{-1} = \sum_{n=0}^{\infty} b_n X^n$, having $b_n = \sum_{k=0}^{n}a_n a'_{n-k}$. We want $b_0 = 1$ and $b_i = 0$ for $i>0$. So clearly let $a'_0 = a_0^{-1}$ to get $b_0=1$.
			
			For $a'_n$ where $n>0$, we construct recursively with
			\[0 = \sum_{k=0}^{n}a_n a'_{n-k} \implies 0 = a_0a'_n + \sum_{k=1}^{n} a_na'_{n-k} \implies a'_n = -a_0^{-1}\sum_{k=1}^{n} a_na'_{n-k}.\]
			
			This is well defined, as every $a'_n$ will be constructed with a sum of finite terms, all of which defined in sequence.
			
			We have now defined every $a'_n$ so that $ff^{-1}=1$, making $f$ a unit in $S$.
		\end{proof}
		
		
		\item Let $S$ be a ring, let $R$ be a subring of $S$, and let $u$ be a fixed element of $S$ which is not in $R$. Consider
		\[ T = \{a+bu : a,b\in R \}\]
		Prove that $T$ is a subring of $S$ if and only if there exists a monic polynomial $f(X) \in R[x]$ of degree 2 with $f(u) = 0$.
		
		\begin{proof}
			$(\implies)$\\
			Assume $T$ is a subring of $S$. Since $R$ is a subring, we have the same zero and one elements, denoting $0$ and $1$, in $R$. So then $u \in T$.
			
			Since $T$ is a subring, it is closed under multiplication. So $u^2 = a'+b'u$ for some $a',b' \in R$.
			
			It is now easy to see that the following monic of degree 2 is equal to 0 when evaluated at $u$.
			\[f(x) = x^2 - b'x-a'\]

			$(\impliedby)$\\
			Assume $f(X) \in R[x]$ is a monic polynomial of degree 2 with $f(u)=0$, denoted $f(x) = x^2 -b'x -a'$ for some $a',b' \in R$. Then we can also say $u^2 = b'u + a'$.
			
			First, it is obvious that every element of $T = \{a+bu : a,b \in R\}$ is also in $S$. It is also clear that $T$ is closed under addition since $R$ is a subring and also closed under addition and multiplication. It is less clear for multiplication.
			
			Let $a_1 +b_1u$ and $a_2 +b_2u$ be elements of $T$. Then 
			\begin{align*}
				(a_1 +b_1u)(a_2 +b_2u) &= (a_1a_2) + (a_1b_2 + b_1a_2)u + b_1b_2u^2\\
				& = (a_1a_2) + (a_1b_2 + b_1a_2)u + b_1b_2(b'u+a')
			\end{align*}
			which will clearly be in $T$ as $R$ is closed under addition and multiplication. So $T$ is closed under multiplication.
			
			Finally, $T$ also contains $1$ since $R$ also contains $1$. So $T$ is a subring of $S$.
		\end{proof}
		
		\item For each of the following, decide whether the set
		\[ T = \{a+bu : a,b\in \Z \}\]
		is a subring of $\R$ or not. Justify your answers. You may use the result from problem 2.
		\begin{enumerate}
			\item $u = 1 + \sqrt{2}$\m
			We have 
			\[ T = \{a+b(1+\sqrt{2}) : a,b\in \Z \} \]
			Check $u^2$:
			\begin{align*}
				(1 + \sqrt{2})^2 = 3 + 2\sqrt{2} = 1 + 2 + 2\sqrt{2} = 1+2(1+\sqrt{2})
			\end{align*}
			So then we could construct a monic degree 2 polynomial with $f(u)=0$, following the same idea as the forward direction in proof of problem 2. So $T$ is a subring.
			
			\item $u = (1 + \sqrt{3})/2$
			We have 
			\[ T = \left\{a + b\left(\frac{1+\sqrt{3}}{2}\right) : a,b\in \Z \right\} \]
			Consider $u \cdot u$:
			\begin{align*}
				\left( \frac{1+\sqrt{3}}{2} \right)^2 = 1 + \frac{\sqrt{3}}{2} = \frac{1}{2} + \left(\frac{1+\sqrt{3}}{2}\right) \nin T
			\end{align*}
			So $T$ is not closed under multiplication, making it not a subring.
		\end{enumerate}
		
		\item Let $R = \{ a+bi : a,b \in \Z\}$. It is easy to check that $R$ is a subring of $\C$ (don't do). Consider the function $\Phi : R \rightarrow \Z$ defined by $\Phi(a+bi) = a^2 + b^2$.
		\begin{enumerate}
			\item Prove that $a+bi$ is a unit in $R$ if and only if $\Phi(a+bi)=1$.
			\begin{proof}
				$(\implies)$\\
				Assume $a+bi$ is a unit in $R$. Denote $c+di$ as its multiplicative inverse.
				\[1 = (a+bi)(c+di) = ac + (ad+bc)i - bd \implies\]
				\[1 = ac - bd \qquad 0 = ad+bc\]
				Seeing $ac-bd$ gives the idea of using matrices. We then can have
				\begin{align*}
					ac-bd &= 1\\
					ad+bc &= 0
				\end{align*}
				giving
				\[\begin{bmatrix}
					a & -b \\
					b & a
				\end{bmatrix}
				\begin{bmatrix}
					c \\
					d
				\end{bmatrix} = 
				\begin{bmatrix}
					1 \\
					0
				\end{bmatrix}\]
				For that system to have $(c,d) \in \Z^2$ as a solution, the determinant of $\begin{bmatrix}
									a & -b \\
									b & a
								\end{bmatrix}$ must be $\pm1$, making sure
				the matrix is invertible over the integers.
				
				So then $\det\left(\begin{bmatrix}
			 					 a & -b \\
								 b & a
							 \end{bmatrix}\right) = a^2 + b^2 = 1$ since squares of integers can never be negative.
				
				
				$(\impliedby)$\\
				Assume $\Phi(a+bi) = 1$. So then $a^2 + b^2 = 1$. So then we have $(a,b) \in \{(\pm1,0),(0,\pm1)\}$ since $a,b \in \Z$. This gives a list of possible values for $a+bi$ being $1,-1,i,-i$.
				
				To check if the elements are units: $1$ and $-1$ are their own multiplicative inverses, while $i$ and $-i$ are multiplicative inverses of each other.
			\end{proof}
			
			\item Use the result from part a. to find all the units in $R$.\m
			The reverse direction of the proof lists outs that the only units are $-1,1,-i,i$.
		\end{enumerate}
	\end{enumerate}
\end{document}