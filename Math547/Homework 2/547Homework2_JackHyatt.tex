% !TeX program = lualatex

\documentclass[12pt]{article}



\usepackage[margin=1in]{geometry} 
\usepackage{amsmath,amsthm,amssymb}
\usepackage{MnSymbol}
\usepackage{graphicx}
\usepackage{bm}
\usepackage[normalem,normalbf]{ulem}
\usepackage{algorithm} 
\usepackage{algpseudocode} 
\usepackage{multirow}
\usepackage{rotating}
\usepackage{therefore}

\usepackage{tikz}
\usetikzlibrary{shapes.multipart}
\usetikzlibrary{shapes.symbols}

\usetikzlibrary{graphs,graphdrawing,graphs.standard,quotes}
\usegdlibrary{circular,force,layered,routing}
\tikzset{
	graphs/simpleer/.style={
		nodes={draw,circle, blue, left color=blue!20, text=black, inner sep=1pt},
		node distance=2.5cm, nodes={minimum size=2em}
	},
	every loop/.style={},
}

\newcommand*\circled[1]{\tikz[baseline=(char.base)]{
		\node[shape=circle,draw,inner sep=2pt] (char) {#1};}}

\newcommand{\m}{\medskip\\}
\newcommand{\N}{\mathbb{N}}
\newcommand{\Z}{\mathbb{Z}}
\newcommand{\R}{\mathbb{R}}
\newcommand{\bbs}{\textbackslash\textbackslash\space}
\newcommand{\bs}{\textbackslash\space}
\newcommand{\la}{\enskip\land\enskip}
\newcommand{\lo}{\enskip\lor\enskip}
\newcommand{\comp}[1]{#1^\mathsf{c}}
\newcommand{\micdrop}{\qed}
\newcommand{\contra}{\begin{tikzpicture}
		\node[starburst, draw, minimum width=3cm, minimum height=2cm,line width=1.5pt,red,fill=yellow,scale=.5]
		{BOOM, A CONTRADICTION!!!};
\end{tikzpicture}}

\renewcommand{\qedsymbol}{$\blacksquare$}

\DeclareMathOperator{\lcm}{lcm}

\newtheorem{theorem}{Theorem}

\newenvironment{exercise}[2][Exercise]{\begin{trivlist}
		\item[\hskip \labelsep {\bfseries #1}\hskip \labelsep {\bfseries #2.}]}{\end{trivlist}}

\setlength\parindent{24pt}

\makeatletter
\renewcommand*\env@matrix[1][*\c@MaxMatrixCols c]{%
	\hskip -\arraycolsep
	\let\@ifnextchar\new@ifnextchar
	\array{#1}}
\makeatother
\setlength\parindent{24pt}


\begin{document}
	
	% --------------------------------------------------------------
	%                         Start here
	% --------------------------------------------------------------
	
	
	\title{Homework 2 (Due Jan 29, 2025)}
	\author{Jack Hyatt\\ %replace with your name
		MATH 547 - Algebraic Structures II - Spring 2025} 
	
	\maketitle
	
	Justify all of your answers completely.\\
	
	
	\medskip 
	
	\begin{enumerate}
		\item Let $F$ be a field, $f(X) \in F[X]$ a polynomial, and $u \in F$ a root of $f$. Prove that $u$ is a multiple root of $f(X)$ if and only if it is also a root of the derivative $f'(x)$.
		\begin{proof}
			For both directions, $u$ is a root of $f$.\\
			$(\implies)$\\
			Assume $u$ is a multiple root of $f(x)$.
			
			Then $f(x) = g(x)(x-u)^k$ with $k\geq2$.
			
			Consider $f'(x)$.
			
			\[f'(x) = g(x)k(x-u)^{k-1} + g'(x)(x-u)^k = (x-u)^{k-1} (kg(x)+g'(x)(x-u))\]
			
			Since $k \geq 2$, we have $k-1\geq1$. So $x-u \divides f'$, making $u$ a root of $f'$.
			
			$(\impliedby)$\\
			Assume $u$ is a root of $f'$ (still is also a root of $f$).
			
			Then $x-u \divides f$, we have $f(x) = g(x)(x-u)$ for some polynomial $g$.
			
			Consider $f'(x)$.
			
			\[f'(x) = g(x) + g'(x)(x-u)\]
			\[0 = f'(u) = g(u) + g'(u)(u-u) \implies g(u) = 0\]
			Since $u$ is a root of $g$, it also divides. So we have $g(u) = h(u)(x-u)$ for some function $h$. So then 
			\[f(x) = g(x)(x-u) = h(x)(x-u)(x-u)\]
			making $u$ a multiple root.
		\end{proof}
		
		\item Let $f(x) = x^5 + 4x^4 + 2x^3 + 3x^2 \in \Z_5[x]$ and $g(x) = x^2 + 3 \in \Z_5[x]$. Find the quotient and the remainder when $f(x)$ is divided by $g(x)$.\m
		First divide and subtract the first terms:
		\[x^5/x^2 = x^3 \implies (x^5 + 4x^4 + 2x^3 + 3x^2) - (x^2 + 3)x^3 = 4x^4 + 4x^3 + 3x^2\]
		Next term:
		\[4x^4/x^2 = 4x^2 \implies (4x^4 + 4x^3 + 3x^2) - (x^2 + 3)4x^2 = 4x^3 + x^2\]
		Next term:
		\[4x^3/x^2 = 4x \implies (4x^3 + x^2) - (x^2 + 3)4x = x^2 + 3x\]
		Next term:
		\[x^2/x^2 = 1 \implies (x^2 + 3x) - (x^2 + 3)1 = 3x + 2\]
		So then we finally have 
		\[x^5 + 4x^4 + 2x^3 + 3x^2 = (x^2 + 3)(x^3 + 4x^2 + 4x + 4) + (3x+2)\]
		
		\item Let $f(x) = x^4 - 5x^2 + 6$. Observe that $f(x)$ can be viewed as a polynomial in $\Q[x]$, or $\R[x]$, or $\Z_p[x]$. A different one of these fields is used in each of the parts below.
		\begin{enumerate}
			\item Find all the roots of $f$ in $\Q$.\m
			Let us act like we are looking for roots in $\R$, and see if the values would land in $\Q$.
			\[f(x) = (x^2)^2 -5(x^2)+6 \implies x^2 = \frac{5 \pm \sqrt{(-5)^2 -4(1)(6)}}{2(1)} = 3,2\]
			\[x = \pm \sqrt{3}, \pm \sqrt{2}\]
			So there are no roots in $\Q$.
			
			\item Find all the roots of $f$ in $\R$.\m
			Found in part (a), $\{-\sqrt{3},-\sqrt{2},\sqrt{2},\sqrt{3}\}$
			
			\item Find all the roots of $f$ in $\Z_3$.\m
			Since $f$ is in $\Z_3[x]$, we can write it as
			\[f(x) = x^4+x^2 = x^2(x^2+1)\]
			which gives roots of just $0$, since no element in $Z_3$ squares to $2$ (i.e. $-1$).
			
			\item Find all the roots of $f$ in $\Z_5$.\m
			Since $f$ is in $\Z_5[x]$, we can write it as
			\[f(x) = x^4+1.\]
			We can just check all 5 elements to look for $x^4 = 4$.
			\[((0)^4,(1)^4,(2)^4,(3)^4,(4)^4) = (0,1,1,1,1)\]
			So we have no roots.
			
		\end{enumerate}
	
		\item Let $f(x) = x^4 + 5x^2 + 6 \in \R[x]$.
		\begin{enumerate}
			\item Prove that $f(x)$ does not have any roots in $\R$.\m
			\[x^2 = \frac{-5 \pm \sqrt{5^2 -4(1)(6)}}{2(1)} = -3,-2\]
			We can stop here, as we know that no elements in $\R$ can square to $-3$ or $-2$. So therefore, $f$ does not have any roots in $\R$.
			
			\item Find all the roots of $f(x)$ in $\C$.\m
			Continuing from part (a), we have
			\[x = \pm \sqrt{-3},\pm \sqrt{-2}\]
			
			\item Factor $f$ into irreducible factors in $\R[x]$. Explain why the factors are irreducible.\m
			Using our work from part (a), we can easily get 
			\[f(x) = (x^2 + 3)(x^2 + 2)\]
			We know these terms are irreducible since they have degree less than or equal to 3, and no roots in $\R$.
			
			\item Factor $f$ into irreducible factors in $\C[x]$. Explain why the factors are irreducible.\m
			\[f(x) = (x-i\sqrt{3})(x+i\sqrt{3})(x+i\sqrt{2})(x-i\sqrt{2})\]
			Since these are linear terms, they are obviously irreducible.
		\end{enumerate}
		
	\end{enumerate}
\end{document}