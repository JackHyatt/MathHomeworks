% !TeX program = lualatex

\documentclass[12pt]{article}



\usepackage[margin=1in]{geometry} 
\usepackage{amsmath,amsthm,amssymb}
\usepackage{MnSymbol}
\usepackage{graphicx}
\usepackage{bm}
\usepackage[normalem,normalbf]{ulem}
\usepackage{algorithm} 
\usepackage{algpseudocode} 
\usepackage{multirow}
\usepackage{rotating}
\usepackage{therefore}

\usepackage{tikz}
\usetikzlibrary{shapes.multipart}
\usetikzlibrary{shapes.symbols}

\usetikzlibrary{graphs,graphdrawing,graphs.standard,quotes}
\usegdlibrary{circular,force,layered,routing}
\tikzset{
	graphs/simpleer/.style={
		nodes={draw,circle, blue, left color=blue!20, text=black, inner sep=1pt},
		node distance=2.5cm, nodes={minimum size=2em}
	},
	every loop/.style={},
}

\newcommand*\circled[1]{\tikz[baseline=(char.base)]{
		\node[shape=circle,draw,inner sep=2pt] (char) {#1};}}

\newcommand{\m}{\medskip\\}
\newcommand{\N}{\mathbb{N}}
\newcommand{\Z}{\mathbb{Z}}
\newcommand{\R}{\mathbb{R}}
\newcommand{\bbs}{\textbackslash\textbackslash\space}
\newcommand{\bs}{\textbackslash\space}
\newcommand{\la}{\enskip\land\enskip}
\newcommand{\lo}{\enskip\lor\enskip}
\newcommand{\comp}[1]{#1^\mathsf{c}}
\newcommand{\micdrop}{\qed}
\newcommand{\contra}{\begin{tikzpicture}
		\node[starburst, draw, minimum width=3cm, minimum height=2cm,line width=1.5pt,red,fill=yellow,scale=.5]
		{BOOM, A CONTRADICTION!!!};
\end{tikzpicture}}

\renewcommand{\qedsymbol}{$\blacksquare$}

\DeclareMathOperator{\lcm}{lcm}

\newtheorem{theorem}{Theorem}

\newenvironment{exercise}[2][Exercise]{\begin{trivlist}
		\item[\hskip \labelsep {\bfseries #1}\hskip \labelsep {\bfseries #2.}]}{\end{trivlist}}

\setlength\parindent{24pt}

\makeatletter
\renewcommand*\env@matrix[1][*\c@MaxMatrixCols c]{%
	\hskip -\arraycolsep
	\let\@ifnextchar\new@ifnextchar
	\array{#1}}
\makeatother
\setlength\parindent{24pt}
\newtheorem{lemma}[theorem]{Lemma}
\usepackage{tikzsymbols}


\begin{document}
	
	% --------------------------------------------------------------
	%                         Start here
	% --------------------------------------------------------------
	
	
	\title{Homework 10 (Due April 28, 2025)}
	\author{Jack Hyatt\\ %replace with your name
		MATH 547 - Algebraic Structures II - Spring 2025} 
	
	\maketitle
	
	Justify all of your answers completely.\\
	
	
	\medskip 
	
	
	\begin{enumerate}
		\item Let $f_n$ denote the $n$th cyclotomic polynomial. Let $E = \Q(z_1)$ be the splitting field of $f_n$ over $\Q$, where $z_1 = e^{\frac{2\pi}{n}}$ and let $G = Gal(E/\Q)$.\\
		We say in class that $G = \{ \sigma_k : 1 \leq k \leq n, \gcd(k,n) = 1\}$, where $\sigma_k$ is the unique $\Q$-automorphism of $E$ with $\sigma_k(z_1) = (z_1)^k$.\\
		Let $\phi : \Z^*_n \rightarrow G$ be defined as $\phi([k]_n) = \sigma_k$.
		\begin{enumerate}
			\item Verify that $\phi$ is a well-defined functioned.\m
			Let $k \equiv \ell \mod n$. Then $xn + \ell = k$ and
			\[z^k_1 = (e^{2i\pi/n})^k = (e^{2i\pi/n})^{xn + \ell} = (e^{2i\pi/n})^{xn + \ell} = (e^{2i\pi/n})^\ell = z^\ell_1\]
			So then $\sigma_k(z_1) = \sigma_\ell(z_1)$. Since $z_1$ generates $E$ over $\Q$, and automorphisms are defined by their mappings of the generators, it implies that $\sigma_k = \sigma_\ell$.
			
			\item Verify that $\phi$ is a group homomorphism.\m
			\[\phi([k]_n \cdot [\ell]_n) = \phi([k\ell]_n) = \sigma_{k\ell} \]
			\[\phi([k]_n)\phi([\ell]_n) = \sigma_k \sigma_\ell = \sigma_{k\ell}\]
			
			\item Verify that $\phi$ is one to one.\m
			Let $\phi([k]_n) = \phi([\ell]_n)$. Then $\sigma_k = \sigma_\ell$, which gives $z^k_1 = z^\ell_1 \implies e^{2ik\pi/n} = e^{2i\ell\pi/n}$. So
			\[2i\pi\frac{k}{n} = 2i\pi\left(\frac{\ell}{n}+x\right) \implies k = \ell +nx \implies k \equiv \ell \mod n\]
			
		\end{enumerate}
		
		\item Let $E$ denote the splitting field of $f(x) = x^3 - 2$ over $\Q$. Recall that $E = \Q(\sqrt[3]{2}, \omega)$ where $\omega$ is a primitive 3rd root of 1, and that $[E : \Q] = 6$. Let $G = \text{Gal}(E/\Q)$.
		\begin{enumerate}
			\item Prove that $G$ is isomorphic to $S_3$ (you may use the facts that were stated in class).
			\begin{proof}
				Since $[E:\Q] = 6$, then $|G| = 6$. We know that $G$ is isomorphic to a subgroup to $S_3$, since it is the Galois group of the splitting field of degree 3. The only subgroup of $S_3$ with order $6$ is $S_3$ itself. So $G$ is isomorphic to $S_3$.
			\end{proof}
		
			\item Find an element of $G$ that has order equal to 3. Describe this element by saying what it does to the two generators of $E$ over $\Q$, and verify that it has order 3.
			\begin{proof}
				Consider the automorphism defined by, $\sigma(\sqrt[3]{2}) = \sqrt[3]{4}$ and $\sigma(\omega)=\omega$. Clearly, it is an element of Gal$(E/\Q)$. Now to check that the order is not 2, which would make the order 3 since $G$ is isomorphic to $S_3$ and that has elements of max degree 3.\\
				\begin{align*}
					\sigma^2(\sqrt[3]{2}) &= \sigma(\sqrt[3]{4})\\
					 &= \sigma((\sqrt[3]{2})^2)\\
					 &= (\sigma(\sqrt[3]{2}))^2\\
					 &= (\sqrt[3]{4})^2\\
					 &= 2\sqrt[3]{2} \neq \sqrt[3]{2}
				\end{align*}
				Since $\sigma^2(\sqrt[3]{2}) \neq \sqrt[3]{2}$, we can immediately say the order is not 2.
			\end{proof}
		\end{enumerate}
	
		\item Using the notation from problem 2., let $H$ denote the subgroup
		of $G$ generated by the element you found in part 2b.\\
		Find the fixed field of $H$, $E^H = \{x \in E : \sigma(x) = x\ \forall \sigma \in H\}$. Give your answer in the form $E = \Q(\ldots)$ (specify what elements are adjoined to $\Q$ in order to get $E$). Prove your answer.\\
		(hint: use a basis for E over $\Q$ to express a general element of $E$ as
		a linear combination of the basis elements. Solve for the coefficients in the linear combination that ensure $\sigma(x) = x$ for $\sigma \in H$).
		\begin{proof}
			Let $H = \{id, \sigma, \sigma^2\}$. Let's express $\sigma$ with how it sends an arbitrary element.
			\[
				\sigma(a + b\sqrt[3]{2} + c\sqrt[3]{4} + d\omega + e\sqrt[3]{2}\omega + f\sqrt[3]{4}\omega) = a + b\sqrt[3]{4} + 2c\sqrt[3]{2} + d\omega + e\sqrt[3]{4}\omega + 2f\sqrt[3]{2}\omega
			\]
			We then end up with the equations:
			\begin{align*}
				a &= a & b &= 2c & e &= 2f\\
				d &= d & c &= b & f &= e
			\end{align*}
			So then we have $b=c=e=f=0$. So $E^H = \{a+d\omega: a,d \in \Q\}$, which clearly equals $\Q(\omega)$.
		\end{proof}
		
		\item The splitting field of $x^4 + 1$ over $\Q$ is $E=\Q(\sqrt{2}, i)$ (you don’t need to prove this) and its Galois group $G$ is isomorphic to the Klein 4-group $\Z_2 \times \Z_2$ (you don’t need to prove this either).
		\begin{enumerate}
			\item How many distinct fields $L$ are there such that $\Q \subseteq L \subseteq E$? Justify your answer.
			\begin{proof}
				By the Fundamental Theorem of Galois Theory, we can just count the number of subgroups of $G$ to find our answer, or subsequently subgroups of $\Z_2 \times \Z_2$.\\
				Every non-identity element (3 of them) of $\Z_2 \times \Z_2$ has degree 2, which means they generate a cyclic subgroup of order 2. So there are 3 subgroups of order 2. Including the entire group and the trivial identity subgroup, there are 5 subgroups of $\Z_2 \times \Z_2$. So there are 5 non-isomorphic intermediary fields.
			\end{proof}
			\item Consider $L = \Q(\sqrt{2}i)$. Write out all the elements of $\text{Gal}(E/L)$ by starting with the 4 elements of $G$ and seeing which ones fix $L$.
			\begin{proof}
				Let us denote elements of $G$ by how they map the generators with tuples (abusing notation \Smiley ). \[G = \{id, \sigma((\sqrt{2},i)) = (-\sqrt{2},i)\}, \tau((\sqrt{2},i)) = (\sqrt{2},-i), \gamma((\sqrt{2},i)) = (-\sqrt{2},-i)\}\]
				Since $L$ is an intermediary extension between $\Q$ and $E$, we only need to check the elements of $G$ for automorphisms that fix $L$.\\
				We only need to check the generators, as the automorphisms are homomorphisms. Clearly $id$ fixes $\sqrt{2}i$.
				\[\sigma(\sqrt{2}i) = \sigma(\sqrt{2})\sigma(i) = -\sqrt{2}i \neq \sqrt{2}i\]
				\[\tau(\sqrt{2}i) = \tau(\sqrt{2})\tau(i) = \sqrt{2}(-i) \neq \sqrt{2}i\]
				\[\gamma(\sqrt{2}i) = \gamma(\sqrt{2})\gamma(i) = -\sqrt{2}(-i) = \sqrt{2}i\]
				So the only elements of $\text{Gal}(E/L)$ are $id$ and $\gamma$.
			\end{proof}
		\end{enumerate}
	\end{enumerate}
\end{document}