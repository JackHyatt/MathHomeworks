% !TeX program = lualatex

\documentclass[12pt]{article}



\usepackage[margin=1in]{geometry} 
\usepackage{amsmath,amsthm,amssymb}
\usepackage{MnSymbol}
\usepackage{graphicx}
\usepackage{bm}
\usepackage[normalem,normalbf]{ulem}
\usepackage{algorithm} 
\usepackage{algpseudocode} 
\usepackage{multirow}
\usepackage{rotating}
\usepackage{therefore}

\usepackage{tikz}
\usetikzlibrary{shapes.multipart}
\usetikzlibrary{shapes.symbols}

\usetikzlibrary{graphs,graphdrawing,graphs.standard,quotes}
\usegdlibrary{circular,force,layered,routing}
\tikzset{
	graphs/simpleer/.style={
		nodes={draw,circle, blue, left color=blue!20, text=black, inner sep=1pt},
		node distance=2.5cm, nodes={minimum size=2em}
	},
	every loop/.style={},
}

\newcommand*\circled[1]{\tikz[baseline=(char.base)]{
		\node[shape=circle,draw,inner sep=2pt] (char) {#1};}}

\newcommand{\m}{\medskip\\}
\newcommand{\N}{\mathbb{N}}
\newcommand{\Z}{\mathbb{Z}}
\newcommand{\R}{\mathbb{R}}
\newcommand{\bbs}{\textbackslash\textbackslash\space}
\newcommand{\bs}{\textbackslash\space}
\newcommand{\la}{\enskip\land\enskip}
\newcommand{\lo}{\enskip\lor\enskip}
\newcommand{\comp}[1]{#1^\mathsf{c}}
\newcommand{\micdrop}{\qed}
\newcommand{\contra}{\begin{tikzpicture}
		\node[starburst, draw, minimum width=3cm, minimum height=2cm,line width=1.5pt,red,fill=yellow,scale=.5]
		{BOOM, A CONTRADICTION!!!};
\end{tikzpicture}}

\renewcommand{\qedsymbol}{$\blacksquare$}

\DeclareMathOperator{\lcm}{lcm}

\newtheorem{theorem}{Theorem}

\newenvironment{exercise}[2][Exercise]{\begin{trivlist}
		\item[\hskip \labelsep {\bfseries #1}\hskip \labelsep {\bfseries #2.}]}{\end{trivlist}}

\setlength\parindent{24pt}

\makeatletter
\renewcommand*\env@matrix[1][*\c@MaxMatrixCols c]{%
	\hskip -\arraycolsep
	\let\@ifnextchar\new@ifnextchar
	\array{#1}}
\makeatother
\setlength\parindent{24pt}


\begin{document}
	
	% --------------------------------------------------------------
	%                         Start here
	% --------------------------------------------------------------
	
	
	\title{Homework 5 (Due Feb 26, 2025)}
	\author{Jack Hyatt\\ %replace with your name
		MATH 547 - Algebraic Structures II - Spring 2025} 
	
	\maketitle
	
	Justify all of your answers completely.\\
	
	
	\medskip 
	
	\begin{enumerate}
		\item Let $R$ be a PID and let $a,b \in R$ not zero and not units. Assume $d = \gcd(a,b)$. Recall this implies that there exist elements $a',b' \in R$ s.t. $a=da'$ and $b=db'$.
		\begin{enumerate}
			\item Prove that $\gcd(a',b') = 1$.
			\begin{proof}
				Let $d'$ be a common divisor of $a'$ and $b'$. Then $a' = d'a''$ and $b' = d'b''$.
				
				Then we have $a = dd'a''$ and $b = dd'b''$. So then $dd''$ is a common divisor of $a$ and $b$. But $d$ is already the greatest common divisor, making $d'$ a unit.
				
				So since the only common divisors of $a'$ and $b'$ are units, then $\gcd(a',b') = 1$.
			\end{proof}
			
			
			\item Let $\ell = a'b'd$ (note that his is equal to $(ab)/d$). Prove that $\ell$ is the least common multiple of $a,b$ (meaning that $a|\ell,b|\ell$, and for any element $L \in R$, if $a|L$ and $b|L$ then $\ell|L$).
			\begin{proof}
				With $a = da'$ and $b = db'$, we easily get $\ell = ab'$ and $\ell = ba'$. So $a|\ell$ and $b|\ell$.
				
				Let $L \in R$ with $a|L$ and $b|L$. So $L = an$ and $L = bm$ for some $n,m \in R$.
				\begin{align*}
					L =& an = da'n \\
					L =& bm = db'm \\
					\implies& a'n = b'm
				\end{align*}
				Since $\gcd(a',b')=1$ , there exists $x,y \in R$ s.t. $a'x + b'y = 1$.
				\begin{align*}
					a'x + b'y &= 1 \\
					n(a'x + b'y) &= n \\
					a'nx + b'ny &= n \\
					b'mx + b'ny &= n \\
					b'(mx + ny) &= n
				\end{align*}
				So then $b' | n$, and it is a similar argument for $a' | m$. So we have $n = b'z$ for some $z\in R$, meaning $L = da'n = da'b'z = \ell z$. So finally we have $\ell|L$.
			\end{proof}
		\end{enumerate}
		
		
		\item 
		\begin{enumerate}
			\item Prove that $5$ is not irreducible as an element of $\Z[i]$.\m
			We know that 5 is not irreducible because we can write $5 = (2-i)\cdot(2+i)$, and it is easy to see that the multiplicative inverse of $2-i$ would need to be $2/5 + i/5 \nin \Z[i]$, and similarly for $2+i$. So neither $2-i$ or $2+i$ are units.

			\item If $p$ is a prime number (i.e. prime as an element of $\Z$) and $p \equiv 3 \mod 4$, prove that $p$ is irreducible as an element of $\Z[i]$.
			\begin{proof}
				BWOC, assume $p = fg$ with $f,g \in \Z[i]$ not units. Take $N(a+bi) = a^2 + b^2$ to be the usual norm for $\Z[i]$. We can take the norm of both sides of $p=fg$ to get
				\begin{align*}
					N(p) &= N(f)N(g)\\
					p^2 &= N(f)N(g)
				\end{align*}
				Since $N(f)$ and $N(g)$ are both positive integers, they must be factors of $p^2$. Since $p$ is prime, the possible factorizations of $p^2$ are limited to $p \cdot p$ or $1 \cdot p^2$. We can ignore the $1 \cdot p^2$ case since that would mean one of the factors, f or g,  is a unit, but we assumed that was not the case.
				
				So assume $N(f) = p$ and $N(g) = p$.\\
				Then $f = a+bi$ for some $a,b\in\Z$, so $N(f) = a^2+b^2 = p$. Since $0^2 \equiv 2^2 \equiv 0 \mod 4$ and $1^2 \equiv 3^2 \equiv 1 \mod 4$, we know that $a^2 \! \mod 4$ and $b^2 \! \mod 4$ must also be either $0$ or $1$. But then $a^2 + b^2 \nequiv 3 \mod 4$, which is a contradiction since $p\equiv3\mod4$. \contra
			\end{proof}
		\end{enumerate}
		
		
		\item Find $\gcd(5,3-i)$ as elements in $\Z[i]$. Prove your answer.
		\begin{proof}
			Let the norm of $\Z[i]$ be the usual $N(a+bi) = a^2+b^2$. Let $d = \gcd(5,3-i)$.
			
			Since $d$ divides $5$, its norm $N(d)$ must divide $N(5)$, which is $N(5)=25$. Similarly, since $d$ also divides $3 - i$, $N(d)$ must also divide $N(3 - i) = 10$.
			
			Thus, $N(d)$ must be a common divisor of $25$ and $10$, giving $\gcd(25, 10) = 5$.
			
			Since norms in $\mathbb{Z}[i]$ must be sums of squares, the possible values for $N(d)$ are either $1$ or $5$.
			
			If $N(d) = 1$, then $d$ is a unit. So let us assume this is not the case.
			
			Then $N(d) = 5$, we solve for integers $a, b$ such that $a^2 + b^2 = 5$.
			
			The integer solutions are:
			\[
			(\pm 2, \pm 1) \quad \text{or} \quad (\pm 1, \pm 2).
			\]
			Thus, possible values for \( d \) are:
			\[
			\pm (2 + i), \quad \pm (2 - i), \quad \pm (1 + 2i), \quad \pm (1 - 2i).
			\]
			
			We will need to only check if $2+i$ and $2-i$ divides both $5$ and $3-i$, as the pairs $2+i$ and $1-2i$, and $2-i$ and $1+2i$ are associates (the unit to multiply with is $i$).
			
			\[(2-i)(7/5 + i/5) = 3 - i\]
			So then we know that $2-i$ is not a divisor of $3-i$ in $\Z[i]$, meaning it isn't a common divisor either.
			
			\[(2+i)(1-i) = 3-i \qquad (2+i)(2-i) = 5 \]
			
			Since $2+i$ is a common divisor and has norm 5, which is the largest possible for a non-unit common divisor, we conclude $\gcd(5, 3 - i) = 2+i$ (and its associates).
		\end{proof}
		
		
		\item Recall that $\Z[i]$ is a PID. Consider the ideal $I = (1+2i, 1+5i)$. Find a generator for $I$. Prove your answer.
		\begin{proof}
			Finding the generator for $I$ is only a matter of finding $\gcd(1+2i, 1+5i)$.
			
			Let the norm of $\Z[i]$ be the usual $N(a+bi) = a^2+b^2$. Let $d = \gcd(1+2i,1+5i)$.
			
			Since $d$ divides $1+2i$, its norm $N(d)$ must divide $N(1+2i)=5$. Similarly, since $d$ also divides $1+5i$, $N(d)$ must also divide $N(1 + 5i) = 26$.
			
			Thus, $N(d)$ must be a common divisor of $5$ and $26$, giving $1$ as the only possibility. So then $d$ is a unit, which means $\gcd(1+2i,1+5i) = 1$. 
			
			So then $I = (1)$, which is the whole ring $\Z[i]$.
		\end{proof}
		
	\end{enumerate}
\end{document}